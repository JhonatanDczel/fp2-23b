\section*{Resumen de clases}

El enfoque de clases sigue la siguiente estructura: 

\textbf{Soldado:}
\begin{lstlisting}[language=Java]
Atributos: nombre, fila, columna, nivelVida, vidaActual, actitud, vive, team.
Métodos: atacar(), defender(), serAtacado(), morir(), getTeam(), setNombre(), setFila(), setColumna(), setNivelVida(), getNombre(), getFila(), getColumna(), getNivelVida(), isLive(), toString(), random(), sumar(), getNumSoldados(), getNumTeam1(), getNumTeam2().
\end{lstlisting}

\textbf{Arquero (hereda de Soldado):}
\begin{lstlisting}[language=Java]
Atributo adicional: numFlechas.
Método adicional: disparar().
\end{lstlisting}

\textbf{Caballero (hereda de Soldado):}
\begin{lstlisting}[language=Java]
Atributos adicionales: armaActual, state.
Métodos adicionales: desmontar(), emvestir(), getArmaActual(), getState().
\end{lstlisting}

\textbf{Espadachin (hereda de Soldado):}
\begin{lstlisting}[language=Java]
Atributo adicional: longitudEspada.
Método adicional: crearMuro().
\end{lstlisting}

\textbf{Lancero (hereda de Soldado):}
\begin{lstlisting}[language=Java]
Atributo adicional: longitudLanza.
Método adicional: schiltrom().
\end{lstlisting}

\textbf{Ejercito:}
\begin{lstlisting}[language=Java]
Atributos: misEspadachines, misCaballero, misArqueros, misLanceros, team, columna, fila, nombre, reino.
Métodos: setFila(), setColumna(), setNombre(), setReino(), getColumna(), getFila(), getNombre(), getTeam(), getReino(), getMisArqueros(), getMisCaballeros(), getMisEspadachines(), getMisLanceros(), constructor con varias sobrecargas, addSoldados(), random(), toString().
\end{lstlisting}

\textbf{Mapa:}
\begin{lstlisting}[language=Java]
Atributos: TERRENOS, table, terreno.
Métodos: getTable(), getTerreno(), random().
\end{lstlisting}

\textbf{VideoJuego:}
\begin{lstlisting}[language=Java]
Métodos: main(), play(), addEjercito(), printTable(), random(), mover(), atacarEjercito(), atacar(), moverSoldados(), atacarSoldados().
\end{lstlisting}
