\section{Clase Soldado}
\begin{itemize}
    \item Los atributos para la clase soldado son:
\end{itemize}
\begin{lstlisting} [language=java]
  private String nombre;
  private int fila;
  private int nivelAtaque = random(5);
  private int nivelDefensa = random(5);
  private int columna;
  private int nivelVida;
  private int vidaActual;
  private int velocidad;
  private String actitud;
  private boolean vive;
  private String team;
\end{lstlisting}
\begin{itemize}
  \item Cada atributo que lo requiere, tiene sus metodos setters y getters para encapsular la información.
    \item Se usan 3 constructores sobrecargados que son los siguientes:
\end{itemize}

\begin{lstlisting}
  public Soldado(String t) {
    team = t;
    velocidad = 0;
    vive = true;
    actitud = "ataque";

  }
  public Soldado(int v, String t) {
    team = t;
    velocidad = v;
    vive = true;
    actitud = "ataque";
  }
  public Soldado(int v, int nV, String t) {
    team = t;
    vive = true;
    velocidad = v;
    nivelVida = nV;
    actitud = "ataque";
  }
\end{lstlisting}

\begin{itemize}
    \item Estos constructores nos servirán cuando vayamos a crear soldados con distintos datos base. 
      \item Adicionalmente tenemos los metodos de accion del soldado:
\end{itemize}
\begin{lstlisting} [language=java]
  public void atacar() {
    actitud = "ofensiva";
  }
  public void defender() {
    actitud = "defensiva";
  }
  public void huir() {
    actitud = "fuga";
    velocidad += 2;
  }
  public void avanzar() {
    velocidad += 1;
  }

  public void serAtacado() {
    vidaActual -= 1;
    if(vidaActual == 0) morir();
  }
  public void morir() { 
    vive = false;
  }

  public void retroceder() {
    if (velocidad > 0) {
      velocidad = 0;
      actitud = "defensiva";
    } else if (velocidad == 0) {
      velocidad = -1; 
    }
  }
\end{lstlisting}

\begin{itemize}
    \item Las acciones cambian los estados de los atributos del soldado, a los que accederemos despues con los metodos accesores:
\end{itemize}
\begin{lstlisting}
  public String getTeam() {
    return team; 
  }

  public void setNombre(String n) {
    nombre = n; 
  }

  public void setFila(int f) {
    fila = f; 
  }

  public void setColumna(int c) {
    columna = c; 
  }

  public void setNivelVida(int p) {
    nivelVida = p; 
  }

  public String getNombre() {
    return nombre; 
  }

  public int getFila() {
    return fila; 
  }

  public int getColumna() {
    return columna; 
  }

  public int getNivelVida() {
    return nivelVida; 
  }

  public int getNivelAtaque() {
    return nivelAtaque; 
  }

  public int getNivelDefensa() {
    return nivelDefensa; 
  }

  public void setNivelAtaque(int n) {
    nivelAtaque = n; 
  }

  public void setNivelDefensa(int n) {
    nivelDefensa = n; 
  }

  public boolean isLive() { 
    return vive; 
  }
\end{lstlisting}
\begin{itemize}
    \item Estos metodos accesores nos servirán para manejar la lógica interna del videojuego
      \item Adicionalmente tenemos algunos metodos auxiliares que usamos en la misma clase:
\end{itemize}

\begin{lstlisting}
  public String toString() {
    return "Nombre: " + nombre +
    " | Ubicacion: " + fila + ", " + columna +
    " | nivelVida: " + nivelVida + 
    " | Estado: " + (vive ? "Vivo" : "Muerto") +
    " | Actitud: "+ actitud +"\n" ;
  }
  private int random(int n) {
    return (int) (Math.random() * n + 1);
  }
\end{lstlisting}
