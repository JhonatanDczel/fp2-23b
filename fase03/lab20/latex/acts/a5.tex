\section{Juego Personalizado}
El método \texttt{juegoPersonalizado} proporciona una interfaz interactiva para que el usuario personalice y gestione los soldados de dos equipos en un juego de estrategia. Cada opción del menú está asociada a una función específica que realiza acciones como crear, eliminar, clonar o modificar soldados, así como visualizar información sobre los mismos y el estado general de los ejércitos. el método también incluye opciones para avanzar al juego o volver al menú principal.
A continuación se detalla su estructura interna:


\subsection{Inicialización de Variables}
Se utiliza un objeto \texttt{Scanner} para la entrada de datos desde la consola. Se crea una matriz llamada \texttt{table} de dimensiones $10 \times 10$ para representar el campo de juego. Se declaran dos \texttt{ArrayList} llamados \texttt{ejercito1} y \texttt{ejercito2} para almacenar los soldados de cada equipo.
\begin{lstlisting}
Scanner sc = new Scanner(System.in);
Soldado[][] table = new Soldado[10][10];
ArrayList<Soldado> ejercito1 = new ArrayList<Soldado>();
ArrayList<Soldado> ejercito2 = new ArrayList<Soldado>();
fillTable(table, ejercito1, ejercito2);
\end{lstlisting}

\subsection{Rellenar la Matriz y Listas de Soldados}
Se invoca el método \texttt{fillTable} para inicializar la matriz y las listas de soldados de ambos equipos.
\begin{lstlisting}
fillTable(table, ejercito1, ejercito2);
\end{lstlisting}

\subsection{Impresión de Soldados Ordenados}
Se imprime en la consola la información de los soldados ordenados en el campo de juego.
\begin{lstlisting}
System.out.println("SOLDADOS ORDENADOS");
printSoladosOrdenados(table, ejercito1, ejercito2);
System.out.println("###################################");
\end{lstlisting}

\subsection{Selección del Equipo a Personalizar}
Se solicita al usuario que ingrese el símbolo del equipo (\texttt{"*"} o \texttt{"\#"}) que desea personalizar. Se determina el \texttt{ArrayList} del equipo seleccionado mediante un operador ternario.
\begin{lstlisting}
System.out.print("Escoge al equipo a personalizar: ");
String team = sc.next();
ArrayList<Soldado> currentE = team.equals("*") ? ejercito1 : ejercito2;
\end{lstlisting}

\subsection{Menú de Opciones}
Se presenta un menú con varias opciones numeradas que permiten al usuario realizar acciones específicas en relación con los soldados y equipos.
\begin{lstlisting}
System.out.println("Escoge una opcion: ");
// ... (Listado de opciones)
int input = sc.nextInt();
\end{lstlisting}

\subsection{Ejecución de Acciones Según la Opción Seleccionada}
Se utiliza una serie de condicionales para determinar la acción a realizar según la opción ingresada por el usuario.
\begin{lstlisting}
if (input == 1) crearSoldado(table, currentE, team);
if (input == 2) eliminarSoldado(table, currentE, team);
// ... (Resto de opciones)
if (input == 10) jugar();
if (input == 11) volver();
\end{lstlisting}

