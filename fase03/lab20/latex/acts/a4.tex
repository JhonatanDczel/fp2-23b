\section{Clase Mapa}

La clase \texttt{Mapa} se encarga de representar el terreno y la disposición de los ejércitos en un tablero. Aquí hay un análisis detallado de la clase:

\begin{lstlisting}[language=Java]
public class Mapa {
  final private String[] TERRENOS = {"bosque", "campo", "montaña", "desierto", "playa"};
  private Ejercito[][] table = new Ejercito[10][10];
  private String terreno = TERRENOS[random(5)];
\end{lstlisting}

\textbf{Atributos:}
\begin{itemize}
  \item \texttt{TERRENOS}: Un array de strings que representa los tipos de terrenos posibles.
  \item \texttt{table}: Una matriz de objetos \texttt{Ejercito} que representa la disposición de los ejércitos en el mapa.
  \item \texttt{terreno}: Una cadena que representa el tipo de terreno del mapa.
\end{itemize}

\textbf{Constructor:}
\begin{itemize}
  \item No hay un constructor explícito en la clase. La elección del tipo de terreno se realiza automáticamente al instanciar un objeto de la clase.
\end{itemize}

\textbf{Métodos:}
\begin{itemize}
  \item \texttt{getTable()}: Retorna la matriz de ejércitos (\texttt{table}).
  \item \texttt{getTerreno()}: Retorna el tipo de terreno actual.
\end{itemize}

\begin{lstlisting}[language=Java]
  private int random(int n) {
    return (int) (Math.random() * n);
  }
\end{lstlisting}

\textbf{Método Auxiliar:}
\begin{itemize}
  \item \texttt{random(int n)}: Un método auxiliar privado que genera un número aleatorio entre 0 (inclusive) y \texttt{n} (exclusive).
\end{itemize}

\textbf{Observaciones:}
\begin{itemize}
  \item La clase \texttt{Mapa} encapsula la información sobre el terreno y la disposición de los ejércitos en un tablero bidimensional.
  \item La elección del tipo de terreno se realiza automáticamente al crear un objeto de la clase, utilizando el método \texttt{random} para seleccionar un índice aleatorio del array \texttt{TERRENOS}.
  \item La matriz \texttt{table} tiene un tamaño fijo de 10x10, y cada celda puede contener un objeto \texttt{Ejercito} o ser \texttt{null} si no hay ningún ejército en esa posición.
  \item Los métodos \texttt{getTable} y \texttt{getTerreno} permiten acceder a la información del mapa desde otras clases.
\end{itemize}

\textbf{Ejemplo de Uso:}
\begin{lstlisting}[language=Java]
Mapa miMapa = new Mapa();
Ejercito[][] matrizEjercitos = miMapa.getTable();
String tipoTerreno = miMapa.getTerreno();
\end{lstlisting}

En este ejemplo, se crea un objeto \texttt{Mapa} que automáticamente selecciona un tipo de terreno. Luego, se puede acceder a la matriz de ejércitos y al tipo de terreno utilizando los métodos proporcionados.
