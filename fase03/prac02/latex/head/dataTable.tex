
	
	\vspace*{10pt}
	
	\begin{center}	
		\fontsize{17}{17} \textbf{ Informe de práctica 02\itemPracticeNumber}
	\end{center}
	\centerline{\textbf{\Large Tema: \itemTheme}}
	\vspace*{0.5cm}	

	\begin{table}[H]
		\begin{tabular}{|x{4.9cm}|x{4.3cm}|x{5.1cm}|}
			\hline 
			\rowcolor{tablebackground}
			\color{white} \textbf{Estudiante} & \color{white}\textbf{Escuela}  & \color{white}\textbf{Asignatura}   \\
			\hline 
			{\itemStudent \par \itemEmail} & \itemSchool & {\itemCourse \par Semestre: \itemSemester \par Código: \itemCourseCode}     \\
			\hline 			
		\end{tabular}
	\end{table}		
	
	\section{Actividades}
\begin{enumerate}
    \item Primero haremos conexión con MariaDB localmente.
    \item Crearemos una base de datos fp2-23b que contendra las tablas y datos proporcionados en clase.
\end{enumerate}
		
	\section{Equipos, materiales y temas utilizados}
	\begin{itemize}
		\item Sistema Operativo ArchCraft GNU Linux 64 bits Kernell
		\item Sistema Operativo Arch GNU Linux 64 bits Kernell
		\item NeoVim
		\item OpenJDK 64-Bit 20.0.1 
		\item MariaDB Java Client 3.3.2
		\item Git 2.43.0
		\item Cuenta en GitHub con el correo institucional.
		\item Programación Orientada a Objetos.
	\end{itemize}
	\section{URL de Repositorio Github}
	\begin{itemize}
            \item URL del Repositorio GitHub para clonar o recuperar.
            \item \url{https://github.com/JhonatanDczel/fp2-23b.git}
            \item URL para la práctica 02 en el Repositorio GitHub.
            \item \itemUrl
	\end{itemize}
