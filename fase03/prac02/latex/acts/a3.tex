\section{Peticiones al servidor MariaDB}
\begin{itemize}
  \item Por lo pronto solo queremos probar el funcionamiento del patron singleton, de hecho lo que presentaremos es sencillo.
\end{itemize}}

\subsubsection{La clase Peticion.java}
\begin{itemize}
  \item Esta clase es la que se encarga de realiza las peticiones la servidor, recibir los resultados e imprimirlos. Praticamente crear objetos de esta clase es realizar mas de una peticion al servidor.
  \item El patron de diseno \textbf{Singleton} nos permite crear solo una instancia de esta clase. Se logra esta funcionalidad haciendo privado su constructor y solo poder crear una instancia de esta clse llamando a un metodo estatico que solo permita crear una instancia.
\end{itemize}}
\begin{lstlisting}[language=Java]
  private static Peticion instance;

  private Peticion() throws SQLException, ClassNotFoundException{}
\end{lstlisting}
\begin{itemize}
  \item El campo instance almacena la unica instancia de la clase que puede ser creada.
  \item El constructor es privador y solo puede ser utilizado por algun metodo de la misma clase
\end{itemize}}

\subsubsection{Variables importantes}
\begin{lstlisting}[language=Java]
  String url = "jdbc:mariadb://localhost/fp2_23b"; 
  String user = "fp2_23b"; 
  String password = "12345678";
\end{lstlisting}
\begin{itemize}
  \item Como se observa, estos son las variables que almacenan los parámetros necesarios para la conexión a la base de datos, como la URL de la base de datos, el nombre de usuario y la contraseña.
\end{itemize}}
