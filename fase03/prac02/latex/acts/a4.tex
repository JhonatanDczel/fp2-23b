
\section{Clase Videojuego}

La clase \texttt{Videojuego} representa la lógica principal del juego. A continuación, se realiza un análisis detallado de su estructura y funcionamiento.

\subsection{Atributos Estáticos}

\begin{lstlisting}[language=Java]
private static final int MAX_SOLDADOS=10;
private static final String SIMBOL_EJERCITO1="Z";
private static final String SIMBOL_EJERCITO2="X";
\end{lstlisting}

\begin{itemize}
  \item \texttt{MAX_SOLDADOS}: Define el tamaño máximo del tablero y se utiliza para la creación de matrices y listas.
  \item \texttt{SIMBOL_EJERCITO1} y \texttt{SIMBOL_EJERCITO2}: Representan los símbolos que identifican a los ejércitos en el tablero.
\end{itemize}

\subsection{Método Principal \texttt{main}}

\begin{lstlisting}[language=Java]
public static void main(String[] args)
\end{lstlisting}

Este método inicia la ejecución del juego. A continuación, se describen los pasos principales realizados:

\begin{enumerate}
  \item \textbf{Selección de Reinos:}
     \begin{lstlisting}[language=Java]
String[] nombresReinos={"Inglaterra", "Francia", "Sacro Imperio", "Castilla-Aragon", "Moros"};
System.out.println("Jugador 1 y 2, escojan un reino(diferentes):");
int select1=sc.nextInt()-1;
int select2=sc.nextInt()-1;
String reinoJugador1=nombresReinos[select1];
String reinoJugador2=nombresReinos[select2];
Mapa mapa=new Mapa(random.nextInt(4));
     \end{lstlisting}

  \item \textbf{Creación de Ejércitos y Soldados:}
     \begin{lstlisting}[language=Java]
Soldado[][] tablero=new Soldado[MAX_SOLDADOS][MAX_SOLDADOS];
ArrayList<Ejercito> reino1=crearReino(mapa, SIMBOL_EJERCITO1, reinoJugador1);
ArrayList<Ejercito> reino2=crearReino(mapa, SIMBOL_EJERCITO2, reinoJugador2);
Soldado s1=reino1.get(0).soldadoMasFuerte();
Soldado s2=reino2.get(0).soldadoMasFuerte();
     \end{lstlisting}

  \item \textbf{Preparación del Tablero y Visualización:}
     \begin{lstlisting}[language=Java]
preTablero(reino1.get(0), tablero);
preTablero(reino2.get(0), tablero);
mapa.mostrarTableroSoldados(tablero);
reino1.get(0).mostrarSoldados();
reino2.get(0).mostrarSoldados();
     \end{lstlisting}

  \item \textbf{Visualización Gráfica con Swing:}
     \begin{lstlisting}[language=Java]
SwingUtilities.invokeLater(() -> {
    new TableroGUI(tablero);
});
     \end{lstlisting}

  \item \textbf{Información Detallada de los Ejércitos:}
     \begin{lstlisting}[language=Java]
System.out.println("\n-----Ejercito Nro1(Z)----");
// ... (información detallada sobre el Ejército 1)

System.out.println("\n-----Ejercito Nro2(X)----");
// ... (información detallada sobre el Ejército 2)
     \end{lstlisting}

  \item \textbf{Resumen y Determinación del Ganador:}
     \begin{lstlisting}[language=Java]
reino1.get(0).resumenEjercito(reinoJugador1);
reino2.get(0).resumenEjercito(reinoJugador2);
determinarGanador(reino1.get(0), reino2.get(0), reinoJugador1, reinoJugador2);
     \end{lstlisting}

  \item \textbf{Pregunta para Continuar Jugando:}
     \begin{lstlisting}[language=Java]
System.out.println("Quieres seguir jugando?(true/false) ");
boolean seguirJugando=sc.nextBoolean();
if(seguirJugando==false)
    jugar=false;
     \end{lstlisting}
\end{enumerate}

\subsection{Métodos Adicionales}

\subsubsection{\texttt{crearReino}}
\begin{lstlisting}[language=Java]
public static ArrayList<Ejercito> crearReino(Mapa mapa, String n, String reinoN)
\end{lstlisting}

Este método crea un ejército y lo coloca en una posición aleatoria del tablero. También agrega el ejército al mapa.

\subsubsection{\texttt{preTablero}}
\begin{lstlisting}[language=Java]
public static void preTablero(Ejercito ejercito, Soldado[][] tablero)
\end{lstlisting}

Este método coloca soldados en posiciones aleatorias del tablero antes de iniciar el juego.

\subsubsection{\texttt{eliminarSoldado}}
\begin{lstlisting}[language=Java]
public static void eliminarSoldado(Soldado eliminar, ArrayList<Soldado> soldados)
\end{lstlisting}

Este método elimina un soldado de la lista de soldados.

\subsubsection{\texttt{promedioVida}}
\begin{lstlisting}[language=Java]
public static double promedioVida(Ejercito ejercito)
\end{lstlisting}

Calcula y devuelve el promedio de vida de los soldados en un ejército.

\subsubsection{\texttt{determinarGanador}}
\begin{lstlisting}[language=Java]
private static void determinarGanador(Ejercito ejercito1, Ejercito ejercito2, String reinoJugador1, String reinoJugador2)
\end{lstlisting}

Determina el ganador basándose en probabilidades aleatorias y la vida total de los ejércitos.

\subsubsection{\texttt{realizarAccion}}
\begin{lstlisting}[language=Java]
public static void realizarAccion(Soldado[][] tablero, Soldado soldado, ArrayList<Soldado> ejercito, ArrayList<Soldado> ejercitoE, Scanner sc)
\end{lstlisting}

Realiza una acción durante el juego, como encontrar y enfrentar a un enemigo en el tablero.

