\section{Peticion a la base de datos desde Java}
\begin{itemize}
  \item Usaremos el driver controlador para Mariadb
  \item Se inicia la conexion cargando JDBC Driver y creando un objeto Connection:
\end{itemize}

\begin{lstlisting}
    String url = "jdbc:mariadb://localhost/fp2_23b"; 
    String user = "fp2_23b"; 
    String password = "12345678";

    Class.forName("org.mariadb.jdbc.Driver");

    Connection connection = DriverManager.getConnection(url, user, password);
\end{lstlisting}

\begin{itemize}
  \item Primero se registra el controlador de la base de datos MariaDB
  \item Y se establece la conexión a la base de datos usando el metodo getConnection de la clase DriverManager
  \item Ahora realizamos la peticion con un objeto Statement y el metodo executeQuery:
\end{itemize}

\begin{lstlisting}

    Statement statement = connection.createStatement();
    String query = "SELECT * FROM vets";
    ResultSet resultSet = statement.executeQuery(query);

    while (resultSet.next()) {
      int id = resultSet.getInt("id");
      String fn = resultSet.getString("first_name");
      String ln = resultSet.getString("last_name");

      System.out.println("ID: " + id + ", First Name: " + fn + " Last Name: " + ln);
    }
    System.out.println();
\end{lstlisting}

\begin{itemize}
  \item Este codigo sencillo crea peticiones a la base de datos y muestra los datos por consola
  \item Posteriormente creamos una clase Peticion que se encargue de manejar las peticiones
\end{itemize}
