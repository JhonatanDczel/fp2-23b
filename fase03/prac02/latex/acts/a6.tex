\section{Peticiones con un solo objeto: Patron Singleton}
\begin{itemize}
  \item Usando el patron de singleton, al ejemplo anteriormente mostrado, optimiza el espacio usado en peticiones grandes
  \item Al usar un unico objeto en cada petición, ahorramos el espacio que generaría instanciar un nuevo objeto por cada petición
  \item Luego de hacer la implementacion del patron singleton, podemos verificar si se está usando un solo objeto, imprimiendo el Hash como se muestra a continuacion:
\end{itemize}

\begin{lstlisting}
    ...

    ID: 3, First Name: Linda Last Name: Douglas
    ID: 4, First Name: Rafael Last Name: Ortega
    ID: 5, First Name: Henry Last Name: Stevens
    ID: 6, First Name: Sharon Last Name: Jenkins
    ID: 7, First Name: Jorge Last Name: god

    1684106402
    1684106402
    1684106402
    1684106402  
\end{lstlisting}

\begin{itemize}
  \item Como vemos, el hash que se imprime en cada peticion es exactamente el mismo
  \item Esto nos indica que hicimos correctamente la implementacion de singleton para la clase Peticion
\end{itemize}
