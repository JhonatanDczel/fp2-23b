\section{La clase Soldado y clases heredadas}

La clase \texttt{Soldado} y sus clases heredadas (\texttt{Arquero}, \texttt{Caballero}, \texttt{Espadachin} y \texttt{Lancero}) representan las entidades que participan en el juego de estrategia. A continuación, se proporciona un análisis detallado de los métodos y funcionalidades de estas clases.

\subsection{Clase \texttt{Soldado}}

La clase \texttt{Soldado} es la clase base que encapsula las características y comportamientos comunes de todos los soldados en el juego. Aquí hay un resumen de sus principales métodos y atributos:

\textbf{Atributos:}
\begin{itemize}
    \item \texttt{nombre}: Nombre del soldado.
    \item \texttt{fila}: Posición en la fila.
    \item \texttt{columna}: Posición en la columna.
    \item \texttt{nivelVida}: Nivel de vida máximo.
    \item \texttt{vidaActual}: Vida actual del soldado.
    \item \texttt{actitud}: Actitud del soldado (ofensiva o defensiva).
    \item \texttt{vive}: Indica si el soldado está vivo.
    \item \texttt{team}: Equipo al que pertenece el soldado.
    \item \texttt{MAX}: Constante con un valor de 0.
    \item \texttt{numSoldados, numTeam1, numTeam2}: Contadores estáticos para el número total de soldados y la cantidad en cada equipo.
\end{itemize}

\textbf{Constructores:}
\begin{itemize}
    \item \texttt{Soldado(String t)}: Constructor que inicializa el equipo y actualiza el contador según el equipo.
    \item \texttt{Soldado(int nV, String t)}: Constructor que también inicializa el nivel de vida.
\end{itemize}

\textbf{Métodos de Ataque y Defensa:}
\begin{itemize}
    \item \texttt{atacar(), defender()}: Cambian la actitud del soldado.
\end{itemize}

\textbf{Métodos de Estado y Vida:}
\begin{itemize}
    \item \texttt{serAtacado()}: Reduce la vida actual y verifica si el soldado muere.
    \item \texttt{morir()}: Actualiza los contadores y establece que el soldado ha muerto.
    \item \texttt{isLive()}: Retorna si el soldado está vivo.
    \item \texttt{toString()}: Retorna una representación en cadena del soldado.
\end{itemize}

\textbf{Métodos de Acceso y Modificación:}
\begin{itemize}
    \item Métodos \texttt{get} y \texttt{set} para acceder y modificar los atributos.
\end{itemize}

\textbf{Métodos Estáticos:}
\begin{itemize}
    \item \texttt{random(int n)}: Genera un número aleatorio entre 1 y $n$.
    \item \texttt{sumar(Soldado s)}: Crea un nuevo soldado sumando los niveles de vida de dos soldados.
    \item \texttt{getNumSoldados(), getNumTeam1(), getNumTeam2()}: Retorna los contadores estáticos.
\end{itemize}

\subsection{Clase \texttt{Arquero}}

La clase \texttt{Arquero} hereda de \texttt{Soldado} e introduce el concepto de flechas. Aquí hay un resumen de sus principales métodos y atributos:

\textbf{Atributos Adicionales:}
\begin{itemize}
    \item \texttt{numFlechas}: Número de flechas disponibles.
\end{itemize}

\textbf{Constructor:}
\begin{itemize}
    \item \texttt{Arquero(int nivelVida, String team)}: Llama al constructor de \texttt{Soldado} y establece el número de flechas.
\end{itemize}

\textbf{Métodos de Ataque Adicionales:}
\begin{itemize}
    \item \texttt{disparar()}: Simula el disparo de una flecha y muestra la cantidad restante.
\end{itemize}

\subsection{Clase \texttt{Caballero}}

La clase \texttt{Caballero} hereda de \texttt{Soldado} e introduce el concepto de armas y estados. Aquí hay un resumen de sus principales métodos y atributos:

\textbf{Atributos Adicionales:}
\begin{itemize}
    \item \texttt{armaActual}: Tipo de arma actual (lanza o espada).
    \item \texttt{state}: Estado actual del caballero (montado o desmontado).
\end{itemize}

\textbf{Constructor:}
\begin{itemize}
    \item \texttt{Caballero(int nivelVida, String team)}: Llama al constructor de \texttt{Soldado} y establece la arma y estado.
\end{itemize}

\textbf{Métodos de Cambio de Estado y Ataque:}
\begin{itemize}
    \item \texttt{desmontar(), emvestir()}: Cambian el estado y simulan un ataque según el estado.
\end{itemize}

\textbf{Métodos de Obtención de Estado y Arma:}
\begin{itemize}
    \item \texttt{getState(), getArmaActual()}: Obtienen el estado y el tipo de arma actual.
\end{itemize}

\subsection{Clase \texttt{Espadachin}}

La clase \texttt{Espadachin} hereda de \texttt{Soldado} e introduce el concepto de longitud de espada y la capacidad de crear un muro. Aquí hay un resumen de sus principales métodos y atributos:

\textbf{Atributos Adicionales:}
\begin{itemize}
    \item \texttt{logitudEspada}: Longitud de la espada del espadachín.
\end{itemize}

\textbf{Constructor:}
\begin{itemize}
    \item \texttt{Espadachin(int nivelVida, String team, int lEspada)}: Llama al constructor de \texttt{Soldado} y establece la longitud de la espada.
\end{itemize}

\textbf{Métodos Adicionales:}
\begin{itemize}
    \item \texttt{crearMuro()}: Simula la creación de un muro de escudos.
\end{itemize}

\subsection{Clase \texttt{Lancero}}

La clase \texttt{Lancero} hereda de \texttt{Soldado} e introduce el concepto de longitud de lanza y la capacidad de formar un schiltrom. Aquí hay un resumen de sus principales métodos y atributos:

\textbf{Atributos Adicionales:}
\begin{itemize}
    \item \texttt{logitudLanza}: Longitud de la lanza del lancero.
\end{itemize}

\textbf{Constructor:}
\begin{itemize}
    \item \texttt{Lancero(int nivelVida, String team, int lLanza)}: Llama al constructor de \texttt{Soldado} y establece la longitud de la lanza.
\end{itemize}

\textbf{Métodos Adicionales:}
\begin{itemize}
    \item \texttt{schiltrom()}: Simula la formación de un schiltrom.
\end{itemize}

\textbf{Observaciones Generales:}
\begin{itemize}
    \item La clase \texttt{Soldado} proporciona funcionalidades esenciales y se utiliza como base para los diferentes tipos de soldados.
    \item Las clases heredadas (\texttt{Arquero}, \texttt{Caballero}, \texttt{Espadachin}, \texttt{Lancero}) extienden las funcionalidades de \texttt{Soldado} y agregan comportamientos específicos para cada tipo de soldado.
    \item Los métodos adicionales en las clases heredadas permiten simular acciones específicas de cada tipo de soldado, como disparar flechas, cambiar de arma o formar formaciones defensivas.
\end{itemize}
