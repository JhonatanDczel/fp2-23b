%package list
\documentclass{article}
\usepackage[top=3cm, bottom=3cm, outer=3cm, inner=3cm]{geometry}
\usepackage{multicol}
\usepackage{graphicx}
\usepackage{url}
%\usepackage{cite}
\usepackage{hyperref}
\usepackage{array}
%\usepackage{multicol}
\newcolumntype{x}[1]{>{\centering\arraybackslash\hspace{0pt}}p{#1}}
\usepackage{natbib}
\usepackage{pdfpages}
\usepackage{multirow}
\usepackage[normalem]{ulem}
\useunder{\uline}{\ul}{}
\usepackage{svg}
\usepackage{xcolor}
\usepackage{listings}
\lstdefinestyle{ascii-tree}{
    literate={├}{|}1 {─}{--}1 {└}{+}1 
  }
\lstset{basicstyle=\ttfamily,
  showstringspaces=false,
  commentstyle=\color{red},
  keywordstyle=\color{blue}
}
%\usepackage{booktabs}
\usepackage{caption}
\usepackage{subcaption}
\usepackage{float}
\usepackage{array}

\newcolumntype{M}[1]{>{\centering\arraybackslash}m{#1}}
\newcolumntype{N}{@{}m{0pt}@{}}


%%%%%%%%%%%%%%%%%%%%%%%%%%%%%%%%%%%%%%%%%%%%%%%%%%%%%%%%%%%%%%%%%%%%%%%%%%%%
%%%%%%%%%%%%%%%%%%%%%%%%%%%%%%%%%%%%%%%%%%%%%%%%%%%%%%%%%%%%%%%%%%%%%%%%%%%%
\newcommand{\itemEmail}{jariasq@unsa.edu.pe}
\newcommand{\itemStudent}{Jhonatan David Arias Quispe}
\newcommand{\itemCourse}{Fundamentos de Programacion 2}
\newcommand{\itemCourseCode}{1701213}
\newcommand{\itemSemester}{II}
\newcommand{\itemUniversity}{Universidad Nacional de San Agustín de Arequipa}
\newcommand{\itemFaculty}{Facultad de Ingeniería de Producción y Servicios}
\newcommand{\itemDepartment}{Departamento Académico de Ingeniería de Sistemas e Informática}
\newcommand{\itemSchool}{Escuela Profesional de Ingeniería de Sistemas}
\newcommand{\itemAcademic}{2023 - B}
\newcommand{\itemInput}{Del 18 Setiembre 2023}
\newcommand{\itemOutput}{Al 20 Setiembre 2023}
\newcommand{\itemPracticeNumber}{01}
\newcommand{\itemTheme}{Arreglos Estandar}
%%%%%%%%%%%%%%%%%%%%%%%%%%%%%%%%%%%%%%%%%%%%%%%%%%%%%%%%%%%%%%%%%%%%%%%%%%%%
%%%%%%%%%%%%%%%%%%%%%%%%%%%%%%%%%%%%%%%%%%%%%%%%%%%%%%%%%%%%%%%%%%%%%%%%%%%%

\usepackage[english,spanish]{babel}
\usepackage[utf8]{inputenc}
\AtBeginDocument{\selectlanguage{spanish}}
\renewcommand{\figurename}{Figura}
\renewcommand{\refname}{Referencias}
\renewcommand{\tablename}{Tabla} %esto no funciona cuando se usa babel
\AtBeginDocument{%
	\renewcommand\tablename{Tabla}
}

\usepackage{fancyhdr}
\pagestyle{fancy}
\fancyhf{}
\setlength{\headheight}{30pt}
\renewcommand{\headrulewidth}{1pt}
\renewcommand{\footrulewidth}{1pt}
\fancyhead[L]{\raisebox{-0.2\height}{\includegraphics[width=3cm]{img/logo_episunsa.png}}}
\fancyhead[C]{\fontsize{7}{7}\selectfont	\itemUniversity \\ \itemFaculty \\ \itemDepartment \\ \itemSchool \\ \textbf{\itemCourse}}
\fancyhead[R]{\raisebox{-0.2\height}{\includegraphics[width=1.2cm]{img/logo_abet}}}
\fancyfoot[L]{Jhonatan David Arias Quispe}
\fancyfoot[C]{\itemCourse}
\fancyfoot[R]{Página \thepage}

% para el codigo fuente
\usepackage{listings}
\usepackage{color, colortbl}
\definecolor{dkgreen}{rgb}{0,0.6,0}
\definecolor{gray}{rgb}{0.5,0.5,0.5}
\definecolor{mauve}{rgb}{0.58,0,0.82}
\definecolor{codebackground}{rgb}{0.95, 0.95, 0.92}
\definecolor{tablebackground}{rgb}{0.8, 0, 0}

\lstset{frame=tb,
	language=bash,
	aboveskip=3mm,
	belowskip=3mm,
	showstringspaces=false,
	columns=flexible,
	basicstyle={\small\ttfamily},
	numbers=none,
	numberstyle=\tiny\color{gray},
	keywordstyle=\color{blue},
	commentstyle=\color{dkgreen},
	stringstyle=\color{mauve},
	breaklines=true,
	breakatwhitespace=true,
	tabsize=3,
	backgroundcolor= \color{codebackground},
}

\begin{document}
	
	\vspace*{10px}
	
	\begin{center}	
		\fontsize{17}{17} \textbf{ Informe de Laboratorio \itemPracticeNumber}
	\end{center}
	\centerline{\textbf{\Large Tema: \itemTheme}}
	%\vspace*{0.5cm}	

	\begin{flushright}
		\begin{tabular}{|M{2.5cm}|N|}
			\hline 
			\rowcolor{tablebackground}
			\color{white} \textbf{Nota}  \\
			\hline 
			     \\[30pt]
			\hline 			
		\end{tabular}
	\end{flushright}	

	\begin{table}[H]
		\begin{tabular}{|x{4.7cm}|x{4.8cm}|x{4.8cm}|}
			\hline 
			\rowcolor{tablebackground}
			\color{white} \textbf{Estudiante} & \color{white}\textbf{Escuela}  & \color{white}\textbf{Asignatura}   \\
			\hline 
			{\itemStudent \par \itemEmail} & \itemSchool & {\itemCourse \par Semestre: \itemSemester \par Código: \itemCourseCode}     \\
			\hline 			
		\end{tabular}
	\end{table}		
	
	\begin{table}[H]
		\begin{tabular}{|x{4.7cm}|x{4.8cm}|x{4.8cm}|}
			\hline 
			\rowcolor{tablebackground}
			\color{white}\textbf{Laboratorio} & \color{white}\textbf{Tema}  & \color{white}\textbf{Duración}   \\
			\hline 
			\itemPracticeNumber & \itemTheme & 04 horas   \\
			\hline 
		\end{tabular}
	\end{table}
	
	\begin{table}[H]
		\begin{tabular}{|x{4.7cm}|x{4.8cm}|x{4.8cm}|}
			\hline 
			\rowcolor{tablebackground}
			\color{white}\textbf{Semestre académico} & \color{white}\textbf{Fecha de inicio}  & \color{white}\textbf{Fecha de entrega}   \\
			\hline 
			\itemAcademic & \itemInput &  \itemOutput  \\
			\hline 
		\end{tabular}
	\end{table}
	
	\section{Tarea}
	\begin{itemize}		
		\item Actividad 1: escribir un programa donde se creen 5 soldados considerando sólo su nombre. Ingresar sus datos y
después mostrarlos.
Restricción: se realizará considerando sólo los conocimientos que se tienen de FP1 y sin utilizar arreglos estándar,
sólo usar variables simples.
		\item Actividad 2: escribir un programa donde se creen 5 soldados considerando su nombre y nivel de vida. Ingresar sus
datos y después mostrarlos.
Restricción: se realizará considerando sólo los conocimientos que se tienen de FP1 y sin utilizar arreglos estándar,
sólo usar variables simples.
		\item Actividad 3: escribir un programa donde se creen 5 soldados considerando sólo su nombre. Ingresar sus datos y
después mostrarlos.
Restricción: aplicar arreglos estándar.
            \item Actividad 4: escribir un programa donde se creen 5 soldados considerando su nombre y nivel de vida. Ingresar sus
datos y después mostrarlos.
Restricción: aplicar arreglos estándar. (Todavía no aplicar arreglo de objetos)
            \item Actividad 5: escribir un programa donde se creen 2 ejércitos, cada uno con un número aleatorio de soldados entre
1 y 5, considerando sólo su nombre. Sus datos se inicializan automáticamente con nombres tales como “Soldado0”,
“Soldado1”, etc. Luego de crear los 2 ejércitos se deben mostrar los datos de todos los soldados de ambos ejércitos
e indicar qué ejército fue el ganador.
Restricción: aplicar arreglos estándar y métodos para inicializar los ejércitos, mostrar ejército y mostrar ejército
ganador. La métrica a aplicar para indicar el ganador es el mayor número de soldados de cada ejército, puede
haber empates. (Todavía no aplicar arreglo de objetos)
	\end{itemize}
		
	\section{Equipos, materiales y temas utilizados}
	\begin{itemize}
		\item Sistema Operativo ArchCraft GNU Linux 64 bits Kernell
		\item NeoVim
		\item OpenJDK 64-Bit 20.0.1 
		\item Git 2.42.0
		\item Cuenta en GitHub con el correo institucional.
		\item Programación Orientada a Objetos.
		\item Creacion de programas con CLI	
	\end{itemize}
	
	\section{URL de Repositorio Github}
	\begin{itemize}
		\item URL del Repositorio GitHub para clonar o recuperar.
		\item \url{https://github.com/JhonatanDczel/fp2-23b.git}
		\item URL para el laboratorio 01 en el Repositorio GitHub.
		\item \url{https://github.com/JhonatanDczel/fp2-23b/tree/main/fase01/lab01}
	\end{itemize}
	
	\section{Actividad 1: 5 soldados con nombre}
	
	\subsection{Creando una clase Soldado}
	\begin{itemize}	
		\item Necesitamos una clase Soldado que tenga como atributo Name
		\item Para ello definimos la clase en java:
	\end{itemize}	
		
	\begin{lstlisting}[language=java,caption={Clase Soldado}][H]
        public class Soldado{
          public String name;
        
          public void setName(String str){
            this.name = str;
          }
          public String getName(){
            return this.name;
          }
        }
	\end{lstlisting}
        \subsection{Creando el archivo VideoJuego}
        \begin{itemize}
            \item Ahora necesitamos establecer sus nombres y mostrarlos en pantalla
            \item Para eso creamos el archivo principal VideoJuego
        \end{itemize}
        
	\begin{lstlisting}[language=java,caption={Creando el archivo VideoJuego.java}][H]
        public class VideoJuego{
          public static void main(String[] args){
            Soldado rambo = new Soldado("Rambo");
            Soldado goomp = new Soldado("Goomp");
            Soldado joel = new Soldado("Joel");
            Soldado mauricio = new Soldado("Mauricio");
            Soldado rogelio = new Soldado("Rogelio");
            
            System.out.println("Escuadron Sistemas:\n" 
            + rambo.getName() + "\n" 
            + goomp.getName() + "\n" 
            + joel.getName() + "\n" 
            + mauricio.getName() + "\n" 
            + rogelio.getName() + "\n");
          }
        }
	\end{lstlisting}	
	
	\subsection{Commits}
	\begin{lstlisting}[language=bash,caption={Primer Commit Creando carpeta/archivo para laboratorio 01}][H]
        commit 6f8fdd7000a3c75fb8409f05e12df7e941cfb387
        Author: JhonatanDczel <jariasq@unsa.edu.pe>
        Date:   Tue Sep 19 13:20:37 2023 -0500
        
            Actividad 1: Cinco soldados con nombres
        
        commit 6f1f099db7c6a8a7e490f9e58a9eb7366681136f
        Author: JhonatanDczel <jariasq@unsa.edu.pe>
        Date:   Mon Sep 18 17:50:42 2023 -0500
        
            Avance Actividad 1
        
        commit 3b6c3c402b581ca43216641e6ab811dad6007017
        Author: JhonatanDczel <jariasq@unsa.edu.pe>
        Date:   Mon Sep 18 08:23:46 2023 -0500
        
            Actividad 1: inicializando archivos
	\end{lstlisting}
	\begin{itemize}
	    \item Registro de los commits con \textbf{git log}
            \item En total fueron 3 commits
	\end{itemize}
    
        \section{Actividad 2: Agregando el nivel de vida}
	\begin{itemize}
		\item Necesitamos agregar otro atributo: life
            \item El item sera de tipo entero para facilitar futuras implementaciones
            \item Adicionalmente necesitaremos configurar los Getters y Setters para los atributos
	\end{itemize}		
        \subsection{Agregando atributos y metodos getters y setters}
        \begin{lstlisting}[language=java, caption={Creando un nuevo atributo para Soldado}]
      public int life;
      public void setLife(int life){
        this.life = life;
      }
    
      //SECCION DE GETERS
    
      public String getName(){
        return this.name;
      }
    
      public int getLife(){
        return this.life;
      }
        \end{lstlisting}
    \subsection{Ejecucion}
    \begin{lstlisting}[language=bash, caption={Ejecucion del codigo}]
        javac VideoJuego.java
        java VideoJuego
        
        Escuadron Sistemas:
        Rambo
        Goomp
        Joel
        Mauricio
        Rogelio

    \end{lstlisting}
    \section{Actividad 3: Permitir al usuario definir los datos}
        \subsection{Entrada del usuario}
        \begin{itemize}
            \item Como no podemos usar arreglos de objetos, usaremos un arreglo de String para almacenar los datos
        \end{itemize}
        \begin{lstlisting}[language=java, caption={Almacenando los nombres usando un ciclo for}]
        String[] names = new String[5];
        for(int i = 0; i < 5; i++){
          System.out.println("\nIngrese el nombre del soldado numero " + (i + 1) + ":");
          names[i] = sc.next();
        }
    
        \end{lstlisting}
        \subsection{Ejecucion}
        \begin{lstlisting}[language=java, caption={Salida por consola}]
        Ingrese el nombre del soldado numero 1:
        Romel
        
        Ingrese el nombre del soldado numero 2:
        Reginald
        
        Ingrese el nombre del soldado numero 3:
        Ricardo
        
        Ingrese el nombre del soldado numero 4:
        Rambo
        
        Ingrese el nombre del soldado numero 5:
        Rinosio

        \end{lstlisting}
        \begin{itemize}
            \item Como primera parte, el juego te pide ingresar el nombre de los soldados
        \end{itemize}
        \begin{lstlisting}[language=java, caption={Muestreo de datos al usuario}]
        =====DATOS DE SOLDADOS=====
        Soldado: Romel
        Nivel de vida: 0
        
        Soldado: Reginald
        Nivel de vida: 0
        
        Soldado: Ricardo
        Nivel de vida: 0
        
        Soldado: Rambo
        Nivel de vida: 0
        
        Soldado: Rinosio
        Nivel de vida: 0
        \end{lstlisting}
        \begin{itemize}
            \item Como no hemos ingresado el nivel de vida de los soldados, es 0
        \end{itemize}

        \section{Actividad 4: Digitando el nombre y nivel de vida}
        \begin{itemize}
            \item Unicamente agregamos otro array que represente el nivel de vida
        \end{itemize}
        \begin{lstlisting}[language=java, caption={Pidiendo al usuario la vida de los soldados}]
        String[] names = new String[5];
        int[] lifes = new int[5];
    
        for(int i = 0; i < 5; i++){
          System.out.println("\nIngrese el nombre del soldado numero " + (i + 1) + ":");
          names[i] = sc.next();
    
          System.out.println("Ingrese el nivel de vida del soldado numero " + (i + 1) + ":");
          lifes[i] = sc.nextInt();
        }        
        \end{lstlisting}
        \begin{itemize}
            \item Ahora agregamos la posibilidad de ver estos datos
        \end{itemize}
        \begin{lstlisting}[language=java, caption={Codigo para salida de datos}]
        System.out.println("\n=====DATOS DE SOLDADOS=====");
        System.out.println("Soldado: " + s1.getName() + " \nNivel de vida: " + s1.getLife() + "\n");
        System.out.println("Soldado: " + s2.getName() + " \nNivel de vida: " + s2.getLife() + "\n");
        System.out.println("Soldado: " + s3.getName() + " \nNivel de vida: " + s3.getLife() + "\n");
        System.out.println("Soldado: " + s4.getName() + " \nNivel de vida: " + s4.getLife() + "\n");
        System.out.println("Soldado: " + s5.getName() + " \nNivel de vida: " + s5.getLife() + "\n");
        \end{lstlisting}
        \subsection{Ejecucion en consola}
        \begin{lstlisting}[language=java, caption={Porcion de ejecicion de VideoJuego.java}]
        Ingrese el nombre del soldado numero 1:
        Ronaldo
        Ingrese el nivel de vida del soldado numero 1:
        12
        
        Ingrese el nombre del soldado numero 3:
        Rias
        Ingrese el nivel de vida del soldado numero 3:
        100
        
        Ingrese el nombre del soldado numero 4:
        Ryan
        Ingrese el nivel de vida del soldado numero 4:
        12
        
        =====DATOS DE SOLDADOS=====
        Soldado: Ronaldo
        Nivel de vida: 12
        
        Soldado: Rias
        Nivel de vida: 100
        
        Soldado: Ryan
        Nivel de vida: 12
        \end{lstlisting}
        \section{Actividad 5: Simulacion de batalla entre dos ejercitos}
        \begin{itemize}
            \item El codigo esta estructurado en metodos, y a cada uno le pertenece un commit
        \end{itemize}
            \subsection{Commit 1: Metodo de inicializacion de ejercitos}
            \begin{lstlisting}[language=java, caption={Metodo initializeArmy()}]
          public static String[] initializeArmy(){
            Random rand = new Random();
            int randNum = rand.nextInt(5) + 1;
            String[] army = new String[randNum];
        
            for(int i = 0; i < randNum; i++){
              army[i] = "Soldado " + (i + 1);
            }
            return army;
          }
            \end{lstlisting}
        \begin{itemize}
            \item Usamos la clase Random para generar un array con un numero al azar
            \item Luego, con un bucle for recorremos el arreglo estableciendo el nombre de los soldados
            
        \end{itemize}
        \subsection{Commit 2: Creacion del metodo display}
        \begin{itemize}
            \item Necesitamos un metodo para mostrar la cantidad de soldados que tiene un ejercito
        \end{itemize}
        \begin{lstlisting}[language=java, caption={Metodo display()}]
          public static void displayArmy(String[] army){
            System.out.println("\n===== Army Soldiers =====");
            for(String soldier : army){
              System.out.println(" " + soldier);
            }
          }
        \end{lstlisting}
        \begin{itemize}
            \item Usamos un bucle for : each para imprimir uno a uno los nombres de los soldados
        \end{itemize}
        \subsection{Commit 3: Implementacion de un metodo para determinar al ganador}
        \begin{itemize}
            \item Para esto necesitamos saber el numero exacto de soldados en cada ejercito:
        \end{itemize}
        \begin{lstlisting}[language=java, caption={Metodo whoWins()}]
          public static String whoWins(String[] army1, String[] army2){
            if (army1.length > army2.length)
              return "\n***** Army 1 is the winner! *****";
        
            if (army2.length > army1.length)
              return "\n***** Army 2 is the winner! *****";
        
            return "\n***** It's a tie. No clear winner. *****";
          }
        \end{lstlisting}
        \begin{itemize}
            \item Lo hacemos usando el metodo .length en nuestros arrays 
            \item \textcolor{red}{Note la forma peculiar de usar los condicionales if, al ser un metodo podemos omitir los 'else'}
        \end{itemize}
        \subsection{Commit 4: Implementacion del metodo main y CLI}
        \subsubsection{Banner de juego}
        \begin{itemize}
            \item Para dar la experiencia de juego interactivo, hacemos un banner de inicio:
        \end{itemize}
        \begin{lstlisting}[language=bash, caption={Banner del juego}]
                    ╔════════════════════════════╗
                    ║    Welcome to the Battle   ║
                    ║       Simulator Game!      ║
                    ╚════════════════════════════╝
                    ***** Prepare for battle! *****

        \end{lstlisting}
        \begin{itemize}
            \item Este arte/estilo de hacer programas por consola se denomina CLI
            \item CLI quiere decir "Command Line Interface", "Interfaz de Linea de Comandos", y en lo personal me gusta mucho
        \end{itemize}
        \subsubsection{Implementacion del main}
        \begin{itemize}
            \item Necesitamos inicializar dos ejercitos, mostrarlos, e imprimir cual de los dos resulto ganador
        \end{itemize}
        \begin{lstlisting}[language=java, caption={Metodo main() de VideoJuego.java}]
          public static void main(String[] args){
            String[] army1 = initializeArmy(); 
            String[] army2 = initializeArmy();
        
            System.out.println("╔════════════════════════════╗");
            System.out.println("║    Welcome to the Battle   ║");
            System.out.println("║       Simulator Game!      ║");
            System.out.println("╚════════════════════════════╝");
            
            System.out.println("\n***** Prepare for battle! *****");
        
            displayArmy(army1);
            displayArmy(army2);
        
            System.out.println(whoWins(army1, army2));
        
          }
        \end{lstlisting}
        \begin{itemize}
            \item Hacemos uso de los metodos anteriormente creados
            \item Ahora pasamos a hacer las pruebas para la optimizacion
        \end{itemize}
        \subsection{Ejecucion}
        \begin{itemize}
            \item Para nuestro caso de simulacion unicamente tenemos que compilar y ejecutar:
        \end{itemize}
        \begin{lstlisting}[language=bash, caption={Ejecucion en la linea de comandos}]
        javac VideoJuego.java
        java VideoJuego
        ╔════════════════════════════╗
        ║    Welcome to the Battle   ║
        ║       Simulator Game!      ║
        ╚════════════════════════════╝
        
        ***** Prepare for battle! *****
        
        ===== Army Soldiers =====
         Soldado 1
         Soldado 2
         Soldado 3
         Soldado 4
        
        ===== Army Soldiers =====
         Soldado 1
         Soldado 2
        
        ***** Army 1 is the winner! *****
        \end{lstlisting}
        \begin{itemize}
            \item Como hemos podido ver, ha ganado el ejercito 1 ya que tiene 2 soldados de mas
            \item Con esto hemos finalizado todas las actividades
        \end{itemize}
	\section{\textcolor{red}{Rúbricas}}
	
	\subsection{\textcolor{red}{Entregable Informe}}
	\begin{table}[H]
		\caption{Tipo de Informe}
		\setlength{\tabcolsep}{0.5em} % for the horizontal padding
		{\renewcommand{\arraystretch}{1.5}% for the vertical padding
		\begin{tabular}{|p{3cm}|p{12cm}|}
			\hline
			\multicolumn{2}{|c|}{\textbf{\textcolor{red}{Informe}}}  \\
			\hline 
			\textbf{\textcolor{red}{Latex}} & \textcolor{blue}{El informe está en formato PDF desde Latex,  con un formato limpio (buena presentación) y facil de leer.}   \\ 
			\hline 
			
			
		\end{tabular}
	}
	\end{table}
	
	\clearpage
	
	\subsection{\textcolor{red}{Rúbrica para el contenido del Informe y demostración}}
	\begin{itemize}			
		\item El alumno debe marcar o dejar en blanco en celdas de la columna \textbf{Checklist} si cumplio con el ítem correspondiente.
		\item El alumno debe autocalificarse en la columna \textbf{Estudiante} de acuerdo a la siguiente tabla:
	
		\begin{table}[ht]
			\caption{Niveles de desempeño}
			\begin{center}
			\begin{tabular}{ccccc}
    			\hline
    			 & \multicolumn{4}{c}{Nivel}\\
    			\cline{1-5}
    			\textbf{Puntos} & Insatisfactorio 25\%& En Proceso 50\% & Satisfactorio 75\% & Sobresaliente 100\%\\
    			\textbf{2.0}&0.5&1.0&1.5&2.0\\
    			\textbf{4.0}&1.0&2.0&3.0&4.0\\
    		\hline
			\end{tabular}
		\end{center}
	\end{table}	
	
	\end{itemize}
	
	\begin{table}[H]
		\caption{Rúbrica para contenido del Informe y demostración}
		\setlength{\tabcolsep}{0.5em} % for the horizontal padding
		{\renewcommand{\arraystretch}{1.5}% for the vertical padding
		%\begin{center}
		\begin{tabular}{|p{2.7cm}|p{7cm}|x{1.3cm}|p{1.2cm}|p{1.5cm}|p{1.1cm}|}
			\hline
    		\multicolumn{2}{|c|}{Contenido y demostración} & Puntos & Checklist & Estudiante & Profesor\\
			\hline
			\textbf{1. GitHub} & Hay enlace URL activo del directorio para el  laboratorio hacia su repositorio GitHub con código fuente terminado y fácil de revisar. &2 &X &2 & \\ 
			\hline
			\textbf{2. Commits} &  Hay capturas de pantalla de los commits más importantes con sus explicaciones detalladas. (El profesor puede preguntar para refrendar calificación). &4 &X &4 & \\ 
			\hline 
			\textbf{3. Código fuente} &  Hay porciones de código fuente importantes con numeración y explicaciones detalladas de sus funciones. &2 &X &2 & \\ 
			\hline 
			\textbf{4. Ejecución} & Se incluyen ejecuciones/pruebas del código fuente  explicadas gradualmente. &2 &X &1.5 & \\ 
			\hline			
			\textbf{5. Pregunta} & Se responde con completitud a la pregunta formulada en la tarea.  (El profesor puede preguntar para refrendar calificación).  &2 &X &2 & \\ 
			\hline	
			\textbf{6. Fechas} & Las fechas de modificación del código fuente estan dentro de los plazos de fecha de entrega establecidos. &2 &X &1.5 & \\ 
			\hline 
			\textbf{7. Ortografía} & El documento no muestra errores ortográficos. &2 &X &1.5 & \\ 
			\hline 
			\textbf{8. Madurez} & El Informe muestra de manera general una evolución de la madurez del código fuente,  explicaciones puntuales pero precisas y un acabado impecable.   (El profesor puede preguntar para refrendar calificación).  &4 &X &3 & \\ 
			\hline
			\multicolumn{2}{|c|}{\textbf{Total}} &20 & &16.5 & \\ 
			\hline
		\end{tabular}
		%\end{center}
		%\label{tab:multicol}
		}
	\end{table}
	
\clearpage
	
%\clearpage
%\bibliographystyle{apalike}
%\bibliographystyle{IEEEtranN}
%\bibliography{bibliography}
			
\end{document}
