%package list
\documentclass{article}
\usepackage[top=3cm, bottom=3cm, outer=3cm, inner=3cm]{geometry}
\usepackage{multicol}
\usepackage{graphicx}
\usepackage{url}
%\usepackage{cite}
\usepackage{hyperref}
\usepackage{array}
%\usepackage{multicol}
\newcolumntype{x}[1]{>{\centering\arraybackslash\hspace{0pt}}p{#1}}
\usepackage{natbib}
\usepackage{pdfpages}
\usepackage{multirow}
\usepackage[normalem]{ulem}
\useunder{\uline}{\ul}{}
\usepackage{svg}
\usepackage{xcolor}
\usepackage{listings}
\lstdefinestyle{ascii-tree}{
    literate={├}{|}1 {─}{--}1 {└}{+}1 
  }
\lstset{basicstyle=\ttfamily,
  showstringspaces=false,
  commentstyle=\color{red},
  keywordstyle=\color{blue}
}
%\usepackage{booktabs}
\usepackage{caption}
\usepackage{subcaption}
\usepackage{float}
\usepackage{array}

\newcolumntype{M}[1]{>{\centering\arraybackslash}m{#1}}
\newcolumntype{N}{@{}m{0pt}@{}}


%%%%%%%%%%%%%%%%%%%%%%%%%%%%%%%%%%%%%%%%%%%%%%%%%%%%%%%%%%%%%%%%%%%%%%%%%%%%
%%%%%%%%%%%%%%%%%%%%%%%%%%%%%%%%%%%%%%%%%%%%%%%%%%%%%%%%%%%%%%%%%%%%%%%%%%%%
\newcommand{\itemEmail}{jariasq@unsa.edu.pe}
\newcommand{\itemStudent}{Jhonatan David Arias Quispe}
\newcommand{\itemCourse}{Fundamentos de Programacion 2}
\newcommand{\itemCourseCode}{1701213}
\newcommand{\itemSemester}{II}
\newcommand{\itemUniversity}{Universidad Nacional de San Agustín de Arequipa}
\newcommand{\itemFaculty}{Facultad de Ingeniería de Producción y Servicios}
\newcommand{\itemDepartment}{Departamento Académico de Ingeniería de Sistemas e Informática}
\newcommand{\itemSchool}{Escuela Profesional de Ingeniería de Sistemas}
\newcommand{\itemAcademic}{2023 - B}
\newcommand{\itemInput}{Del 18 Setiembre 2023}
\newcommand{\itemOutput}{Al 20 Setiembre 2023}
\newcommand{\itemPracticeNumber}{02}
\newcommand{\itemTheme}{Arreglos Estandar}
%%%%%%%%%%%%%%%%%%%%%%%%%%%%%%%%%%%%%%%%%%%%%%%%%%%%%%%%%%%%%%%%%%%%%%%%%%%%
%%%%%%%%%%%%%%%%%%%%%%%%%%%%%%%%%%%%%%%%%%%%%%%%%%%%%%%%%%%%%%%%%%%%%%%%%%%%

\usepackage[english,spanish]{babel}
\usepackage[utf8]{inputenc}
\AtBeginDocument{\selectlanguage{spanish}}
\renewcommand{\figurename}{Figura}
\renewcommand{\refname}{Referencias}
\renewcommand{\tablename}{Tabla} %esto no funciona cuando se usa babel
\AtBeginDocument{%
	\renewcommand\tablename{Tabla}
}

\usepackage{fancyhdr}
\pagestyle{fancy}
\fancyhf{}
\setlength{\headheight}{30pt}
\renewcommand{\headrulewidth}{1pt}
\renewcommand{\footrulewidth}{1pt}
\fancyhead[L]{\raisebox{-0.2\height}{\includegraphics[width=3cm]{img/logo_episunsa.png}}}
\fancyhead[C]{\fontsize{7}{7}\selectfont	\itemUniversity \\ \itemFaculty \\ \itemDepartment \\ \itemSchool \\ \textbf{\itemCourse}}
\fancyhead[R]{\raisebox{-0.2\height}{\includegraphics[width=1.2cm]{img/logo_abet}}}
\fancyfoot[L]{Jhonatan David Arias Quispe}
\fancyfoot[C]{\itemCourse}
\fancyfoot[R]{Página \thepage}

% para el codigo fuente
\usepackage{listings}
\usepackage{color, colortbl}
\definecolor{dkgreen}{rgb}{0,0.6,0}
\definecolor{gray}{rgb}{0.5,0.5,0.5}
\definecolor{mauve}{rgb}{0.58,0,0.82}
\definecolor{codebackground}{rgb}{0.95, 0.95, 0.92}
\definecolor{tablebackground}{rgb}{0.8, 0, 0}

\lstset{frame=tb,
	language=bash,
	aboveskip=3mm,
	belowskip=3mm,
	showstringspaces=false,
	columns=flexible,
	basicstyle={\small\ttfamily},
	numbers=none,
	numberstyle=\tiny\color{gray},
	keywordstyle=\color{blue},
	commentstyle=\color{dkgreen},
	stringstyle=\color{mauve},
	breaklines=true,
	breakatwhitespace=true,
	tabsize=3,
	backgroundcolor= \color{codebackground},
}

\begin{document}
	
	\vspace*{10px}
	
	\begin{center}	
		\fontsize{17}{17} \textbf{ Informe de Laboratorio \itemPracticeNumber}
	\end{center}
	\centerline{\textbf{\Large Tema: \itemTheme}}
	%\vspace*{0.5cm}	

	\begin{flushright}
		\begin{tabular}{|M{2.5cm}|N|}
			\hline 
			\rowcolor{tablebackground}
			\color{white} \textbf{Nota}  \\
			\hline 
			     \\[30pt]
			\hline 			
		\end{tabular}
	\end{flushright}	

	\begin{table}[H]
		\begin{tabular}{|x{4.7cm}|x{4.8cm}|x{4.8cm}|}
			\hline 
			\rowcolor{tablebackground}
			\color{white} \textbf{Estudiante} & \color{white}\textbf{Escuela}  & \color{white}\textbf{Asignatura}   \\
			\hline 
			{\itemStudent \par \itemEmail} & \itemSchool & {\itemCourse \par Semestre: \itemSemester \par Código: \itemCourseCode}     \\
			\hline 			
		\end{tabular}
	\end{table}		
	
	\begin{table}[H]
		\begin{tabular}{|x{4.7cm}|x{4.8cm}|x{4.8cm}|}
			\hline 
			\rowcolor{tablebackground}
			\color{white}\textbf{Laboratorio} & \color{white}\textbf{Tema}  & \color{white}\textbf{Duración}   \\
			\hline 
			\itemPracticeNumber & \itemTheme & 04 horas   \\
			\hline 
		\end{tabular}
	\end{table}
	
	\begin{table}[H]
		\begin{tabular}{|x{4.7cm}|x{4.8cm}|x{4.8cm}|}
			\hline 
			\rowcolor{tablebackground}
			\color{white}\textbf{Semestre académico} & \color{white}\textbf{Fecha de inicio}  & \color{white}\textbf{Fecha de entrega}   \\
			\hline 
			\itemAcademic & \itemInput &  \itemOutput  \\
			\hline 
		\end{tabular}
	\end{table}
	
	\section{Tarea}
	\begin{itemize}		
		\item En este ejercicio se le solicita a usted implementar el juego del ahorcado utilizando el código parcial que
se le entrega.
Deberá considerar que:
\begin{itemize}
    \item El juego valida el ingreso de letras solamente.
    \item  En caso el usuario ingrese un carácter equivocado le dará el mensaje de error y volverá a solicitar el ingreso
    \item El juego supone que el usuario no ingresa una letra ingresada previamente
    \item El método ingreseLetra() debe ser modificado para incluir las consideraciones de validación
    \item Puede crear métodos adicionales
\end{itemize}

	\end{itemize}
		
	\section{Equipos, materiales y temas utilizados}
	\begin{itemize}
		\item Sistema Operativo ArchCraft GNU Linux 64 bits Kernell
		\item NeoVim
		\item OpenJDK 64-Bit 20.0.1 
		\item Git 2.42.0
		\item Cuenta en GitHub con el correo institucional.
		\item Programación Orientada a Objetos.
		\item Creacion de programas con CLI	
	\end{itemize}
	
	\section{URL de Repositorio Github}
	\begin{itemize}
		\item URL del Repositorio GitHub para clonar o recuperar.
		\item \url{https://github.com/JhonatanDczel/fp2-23b.git}
		\item URL para el laboratorio 02 en el Repositorio GitHub.
		\item \url{https://github.com/JhonatanDczel/fp2-23b/tree/main/fase01/lab02}
	\end{itemize}
	\section{Commit Principal}
	\begin{lstlisting}[language=bash,caption={Commit principal, sacado de git log}][H]
        commit ac4b27bed13d9852c0bd8796cf1562711052890d (HEAD -> main, origin/main)
        Author: JhonatanDczel <jariasq@unsa.edu.pe>
        Date:   Wed Sep 20 16:47:43 2023 -0500
        
            Laboratorio 2: El juego del ahorcado

	\end{lstlisting}
	\section{Metodo getPalabraSecreta}
	
	\subsection{Creacion del metodo}
	\begin{itemize}	
		\item Necesitamos un metodo que tenga como funcion la de escoger un numero al azar y generar con ello un palabra al azar como palabra secreta
		\item Para eso debera recibir un arreglo de Strings como entrada y devolver un String unico como salida
	\end{itemize}	
		
	\begin{lstlisting}[language=java,caption={Metodo getPalabraSecreta}][H]
        public static String getPalabraSecreta(String[] lasPalabras){
            int ind;
            int indiceMayor = lasPalabras.length - 1;
            int indiceMenor = 0;
            ind = (int) (Math.random() * (indiceMayor - indiceMenor + 1)) + indiceMenor;
            return lasPalabras[ind];
        }
	\end{lstlisting}

    
        \section{Metodo mostrarPalabra}
	\begin{itemize}
		\item Dibujar el contenido de un arreglo que contiene nuestra palabra
            \item Lo haremos con un simple bucle for each, para recorrer el arreglo mientras imprimimos el valor
            \item Adicionalmente los separaremos por espacios " " para darle voluimen a nuestra palabra
	\end{itemize}		
        \begin{lstlisting}[language=java, caption={Creando un nuevo atributo para Soldado}]
        public static void mostrarPalabra(char[] palabra){
            for(char c : palabra){
                System.out.print(c + " ");
            }
            System.out.println();
        }
        \end{lstlisting}
    \subsection{Ejecucion}
    \begin{lstlisting}[language=bash, caption={Ejecucion del codigo}]
        a
        _ _ _ _ _ a _ a _ _ _ _
    \end{lstlisting}
    \begin{itemize}
        \item Asi es como se ve la ejecucion en consola, con un ejemplo generico
    \end{itemize}
    
    \section{Metodo ingreseLetra}
        \subsection{Las consideraciones}
        \begin{itemize}
            \item Necesitamos verificar la entrada del usuario para asegurarnos de que sea un solo caracter y que sea una letra
            \item Para eso, nos apoyamos de un metodo auxiliar, para verificar si la entrada es una letra:
        \end{itemize}
        \begin{lstlisting}[language=java, caption={metodo checker()}]
        public static boolean checker(String letra){
            String[] abc = {"a", "b", "c", "d", "e", "f", "g", "h", "i", "j", "k", "l", "m", "n", "o", "p", "q", "r", "s", "t", "u", "v", "w", "x", "y", "z"};
            for(String letraAbc : abc){
                if(letraAbc.equals(letra)){
                    return false;
                }
            }
            return true;
        }
    
        \end{lstlisting}
        \subsection{La condicion de salida}
        \begin{lstlisting}[language=java, caption={Metodo ingreseLetra}]
        public static String ingreseLetra(){
            String laLetra;
            Scanner sc = new Scanner(System.in);
            System.out.println("Ingrese letra: ");
            laLetra = sc.next().toLowerCase();
            while (laLetra.length() != 1 || checker(laLetra)){
                System.out.println("Ingrese una sola letra: ");
                laLetra = sc.next().toLowerCase();
            }
            return laLetra;
        }

        \end{lstlisting}
        \begin{itemize}
            \item Como podemos ver, el metodo se asegura de permitir unicamente el paso de letras
            \item Para trabajar sin distinciones entre mayusculas y minusculas usamos el metodo toLowerCase, para trabajar estandarizadamente y evitar conflictos
        \end{itemize}

        \section{Metodo letraEnPalabraSecreta}
        \begin{itemize}
            \item Ahora necesitamos saber si la letra que ingreso el usuario esta contenida en la palabra secreta
        \end{itemize}
        \begin{lstlisting}[language=java, caption={Implementacion del metodo letraEnPalabraSecreta}]
        public static boolean letraEnPalabraSecreta(String letra, String palSecreta, char[] palabraMostrada){
            boolean letraAdivinada = false;
            for(int i = 0; i < palSecreta.length(); i++){
                if(palSecreta.charAt(i) == letra.charAt(0)){
                    if(palabraMostrada[i] == '_'){
                        palabraMostrada[i] = letra.charAt(0);
                        letraAdivinada = true;
                    }
                }
            }
            return letraAdivinada;
        }       
        \end{lstlisting}
        \begin{itemize}
            \item Recorremos la palabra secreta en busca de la letra que el usuario ingreso, de ser ese el caso, se retorna la letra adivinada.
        \end{itemize}

        \section{Metodo contarLetrasRestantes}
        \begin{itemize}
            \item A este punto necesitamos un metodo que se encargue de verificar cuantas letras quedan por descubrir, para poder finalizat el programa en caso de llegar a 0
        \end{itemize}
            \begin{lstlisting}[language=java, caption={Metodo contarLetrasRestantes}]
        public static int contarLetrasRestantes(char[] palabraMostrada){
            int contador = 0;
            for(char c : palabraMostrada){
                if (c == '_'){
                    contador++;
                }
            }
            return contador;
        }
            \end{lstlisting}
        \begin{itemize}
            \item Recorremos el arreglo en busca de caracteres '-', en caso de encontrarlo, el contador sube
            \item Al final, el metodo devuelve el numero de letras que quedan por adivinar
            
        \end{itemize}
        
        \section{Construccion del metodo main}
        \begin{itemize}
            \item Ahora necesitamos armar todos los componentes para crear el juego, empezamos inicializando variables
        \end{itemize}
        \begin{lstlisting}[language=java, caption={Metodo display()}]
        int contador = 0;
        String letra;
        String[] palabras = { "programacion", "java", "indentacion", "clases", "objetos", "desarrollador", "pruebas", "test", "pruebita"};
        String palSecreta = getPalabraSecreta(palabras);
        int letrasRestantes = palSecreta.length();

        System.out.println(figuras[contador]);
        char[] palabraMostrada = new char[palSecreta.length()];
        for (int i = 0; i < palabraMostrada.length; i++){
            palabraMostrada[i] = '_';
        }
        mostrarPalabra(palabraMostrada);
        \end{lstlisting}
        \begin{itemize}
            \item La variable contador, sera quien nos indique cuantas veces el usuario ha ingresado una respuesta erronea
            \item Entre otras inicializaciones, tenemos la de palabraMostrada, que rellenamos de caracteres '-' haciendo uso de un ciclo for
        \end{itemize}
        \subsection{Estructura del juego}
        \begin{itemize}
            \item Todo el juego estara contenido en un ciclo while, cuya condicion de salida sera que el contador sea menor que 6 y que queden mas de 0 letras por adivinar
        \end{itemize}
        \begin{lstlisting}[language=java, caption={Metodo principal}]
        
        while(contador < 6 && letrasRestantes > 0){
            letra = ingreseLetra();
            char letraChar = letra.charAt(0);
            if(letrasAdivinadas[letraChar - 'a']){
                System.out.println("Ya has ingresado la letra " + letra);
            }else{
                letrasAdivinadas[letraChar - 'a'] = true;
                if(letraEnPalabraSecreta(letra, palSecreta, palabraMostrada)){
                    mostrarPalabra(palabraMostrada);
                    letrasRestantes = contarLetrasRestantes(palabraMostrada);
                }else{
                    contador++;
                    System.out.println(figuras[contador]);
                }
            }
        }
        \end{lstlisting}
        \begin{itemize}
            \item El juego se mantendra en marcha mientras no se hayan alcanzado el llimite de jugadas erroneas, o no se hayan terminado las letras pr adivinar, es decir, adivino la palabra
        \end{itemize}
    \section{Ejecucion}
    \begin{itemize}
        \item Tendremos ahora una demostracion sobre el modo de juego del Ahorcado:
    \end{itemize}

        \begin{lstlisting}[language=bash, caption={Ejecucion en la linea de comandos}]
         +---+
         |   |
             |
             |
             |
             |
        =========
        _ _ _ _
        Ingrese letra:
        e
        _ e _ _
        Ingrese letra:
        t
        t e _ t
        Ingrese letra:
        s
        t e s t
        
        ¡Ganaste! La palabra secreta es: test

        \end{lstlisting}
        \begin{itemize}
            \item Como hemos podido ver, hemos acertado la palabra y parece estar correcto el sistema de deteccion de la palabra
            \item Ahora haremos un test sobre las limitaciones con la entrada de datos
        \end{itemize}
        \begin{lstlisting}[language=bash, caption={Ejecucion por terminal del juego del Ahorcado}]
         +---+
         |   |
             |
             |
             |
             |
        =========
        _ _ _ _ _ _ _
        Ingrese letra:
        3
        Ingrese una sola letra:
        3
        Ingrese una sola letra:
        f
         +---+
         |   |
         O   |
             |
             |
             |
        =========
        Ingrese letra:
        f
        Ya has ingresado la letra f
        Ingrese letra:
        \end{lstlisting}
        \begin{itemize}
            \item Como vemos, la deteccion de entradas incorrectas esta funcionando bien
            \item Se probo con numeros como entrada, y con letras repetidas
            \item Si no se ingresan entradas legales, el programa seguira indefinidamente hasta que se ingrese una jugada legal
        \end{itemize}
	\section{\textcolor{red}{Rúbricas}}
	
	\subsection{\textcolor{red}{Entregable Informe}}
	\begin{table}[H]
		\caption{Tipo de Informe}
		\setlength{\tabcolsep}{0.5em} % for the horizontal padding
		{\renewcommand{\arraystretch}{1.5}% for the vertical padding
		\begin{tabular}{|p{3cm}|p{12cm}|}
			\hline
			\multicolumn{2}{|c|}{\textbf{\textcolor{red}{Informe}}}  \\
			\hline 
			\textbf{\textcolor{red}{Latex}} & \textcolor{blue}{El informe está en formato PDF desde Latex,  con un formato limpio (buena presentación) y facil de leer.}   \\ 
			\hline 
			
			
		\end{tabular}
	}
	\end{table}
	
	\clearpage
	
	\subsection{\textcolor{red}{Rúbrica para el contenido del Informe y demostración}}
	\begin{itemize}			
		\item El alumno debe marcar o dejar en blanco en celdas de la columna \textbf{Checklist} si cumplio con el ítem correspondiente.
		\item El alumno debe autocalificarse en la columna \textbf{Estudiante} de acuerdo a la siguiente tabla:
	
		\begin{table}[ht]
			\caption{Niveles de desempeño}
			\begin{center}
			\begin{tabular}{ccccc}
    			\hline
    			 & \multicolumn{4}{c}{Nivel}\\
    			\cline{1-5}
    			\textbf{Puntos} & Insatisfactorio 25\%& En Proceso 50\% & Satisfactorio 75\% & Sobresaliente 100\%\\
    			\textbf{2.0}&0.5&1.0&1.5&2.0\\
    			\textbf{4.0}&1.0&2.0&3.0&4.0\\
    		\hline
			\end{tabular}
		\end{center}
	\end{table}	
	
	\end{itemize}
	
	\begin{table}[H]
		\caption{Rúbrica para contenido del Informe y demostración}
		\setlength{\tabcolsep}{0.5em} % for the horizontal padding
		{\renewcommand{\arraystretch}{1.5}% for the vertical padding
		%\begin{center}
		\begin{tabular}{|p{2.7cm}|p{7cm}|x{1.3cm}|p{1.2cm}|p{1.5cm}|p{1.1cm}|}
			\hline
    		\multicolumn{2}{|c|}{Contenido y demostración} & Puntos & Checklist & Estudiante & Profesor\\
			\hline
			\textbf{1. GitHub} & Hay enlace URL activo del directorio para el  laboratorio hacia su repositorio GitHub con código fuente terminado y fácil de revisar. &2 &X &2 & \\ 
			\hline
			\textbf{2. Commits} &  Hay capturas de pantalla de los commits más importantes con sus explicaciones detalladas. (El profesor puede preguntar para refrendar calificación). &4 &X &4 & \\ 
			\hline 
			\textbf{3. Código fuente} &  Hay porciones de código fuente importantes con numeración y explicaciones detalladas de sus funciones. &2 &X &2 & \\ 
			\hline 
			\textbf{4. Ejecución} & Se incluyen ejecuciones/pruebas del código fuente  explicadas gradualmente. &2 &X &1.5 & \\ 
			\hline			
			\textbf{5. Pregunta} & Se responde con completitud a la pregunta formulada en la tarea.  (El profesor puede preguntar para refrendar calificación).  &2 &X &2 & \\ 
			\hline	
			\textbf{6. Fechas} & Las fechas de modificación del código fuente estan dentro de los plazos de fecha de entrega establecidos. &2 &X &1.5 & \\ 
			\hline 
			\textbf{7. Ortografía} & El documento no muestra errores ortográficos. &2 &X &1 & \\ 
			\hline 
			\textbf{8. Madurez} & El Informe muestra de manera general una evolución de la madurez del código fuente,  explicaciones puntuales pero precisas y un acabado impecable.   (El profesor puede preguntar para refrendar calificación).  &4 &X &3 & \\ 
			\hline
			\multicolumn{2}{|c|}{\textbf{Total}} &20 & &16 & \\ 
			\hline
		\end{tabular}
		%\end{center}
		%\label{tab:multicol}
		}
	\end{table}
	
\clearpage
	
%\clearpage
%\bibliographystyle{apalike}
%\bibliographystyle{IEEEtranN}
%\bibliography{bibliography}
			
\end{document}
