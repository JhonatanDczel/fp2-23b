%package list
\documentclass{article}
\usepackage[top=3cm, bottom=3cm, outer=3cm, inner=3cm]{geometry}
\usepackage{multicol}
\usepackage{graphicx}
\usepackage{url}
%\usepackage{cite}
\usepackage{hyperref}
\usepackage{array}
%\usepackage{multicol}
\newcolumntype{x}[1]{>{\centering\arraybackslash\hspace{0pt}}p{#1}}
\usepackage{natbib}
\usepackage{pdfpages}
\usepackage{multirow}
\usepackage[normalem]{ulem}
\useunder{\uline}{\ul}{}
\usepackage{svg}
\usepackage{xcolor}
\usepackage{listings}
\lstdefinestyle{ascii-tree}{
    literate={├}{|}1 {─}{--}1 {└}{+}1 
  }
\lstset{basicstyle=\ttfamily,
  showstringspaces=false,
  commentstyle=\color{red},
  keywordstyle=\color{blue}
}
%\usepackage{booktabs}
\usepackage{caption}
\usepackage{subcaption}
\usepackage{float}
\usepackage{array}

\newcolumntype{M}[1]{>{\centering\arraybackslash}m{#1}}
\newcolumntype{N}{@{}m{0pt}@{}}


%%%%%%%%%%%%%%%%%%%%%%%%%%%%%%%%%%%%%%%%%%%%%%%%%%%%%%%%%%%%%%%%%%%%%%%%%%%%
%%%%%%%%%%%%%%%%%%%%%%%%%%%%%%%%%%%%%%%%%%%%%%%%%%%%%%%%%%%%%%%%%%%%%%%%%%%%
\newcommand{\itemEmail}{jariasq@unsa.edu.pe}
\newcommand{\itemStudent}{Jhonatan David Arias Quispe}
\newcommand{\itemCourse}{Fundamentos de Programacion 2}
\newcommand{\itemCourseCode}{1701213}
\newcommand{\itemSemester}{II}
\newcommand{\itemUniversity}{Universidad Nacional de San Agustín de Arequipa}
\newcommand{\itemFaculty}{Facultad de Ingeniería de Producción y Servicios}
\newcommand{\itemDepartment}{Departamento Académico de Ingeniería de Sistemas e Informática}
\newcommand{\itemSchool}{Escuela Profesional de Ingeniería de Sistemas}
\newcommand{\itemAcademic}{2023 - B}
\newcommand{\itemInput}{Del 20 Setiembre 2023}
\newcommand{\itemOutput}{Al 28 Setiembre 2023}
\newcommand{\itemPracticeNumber}{03}
\newcommand{\itemTheme}{Arreglos de Objetos}
%%%%%%%%%%%%%%%%%%%%%%%%%%%%%%%%%%%%%%%%%%%%%%%%%%%%%%%%%%%%%%%%%%%%%%%%%%%%
%%%%%%%%%%%%%%%%%%%%%%%%%%%%%%%%%%%%%%%%%%%%%%%%%%%%%%%%%%%%%%%%%%%%%%%%%%%%

\usepackage[english,spanish]{babel}
\usepackage[utf8]{inputenc}
\AtBeginDocument{\selectlanguage{spanish}}
\renewcommand{\figurename}{Figura}
\renewcommand{\refname}{Referencias}
\renewcommand{\tablename}{Tabla} %esto no funciona cuando se usa babel
\AtBeginDocument{%
	\renewcommand\tablename{Tabla}
}

\usepackage{fancyhdr}
\pagestyle{fancy}
\fancyhf{}
\setlength{\headheight}{30pt}
\renewcommand{\headrulewidth}{1pt}
\renewcommand{\footrulewidth}{1pt}
\fancyhead[L]{\raisebox{-0.2\height}{\includegraphics[width=3cm]{img/logo_episunsa.png}}}
\fancyhead[C]{\fontsize{7}{7}\selectfont	\itemUniversity \\ \itemFaculty \\ \itemDepartment \\ \itemSchool \\ \textbf{\itemCourse}}
\fancyhead[R]{\raisebox{-0.2\height}{\includegraphics[width=1.2cm]{img/logo_abet}}}
\fancyfoot[L]{Jhonatan David Arias Quispe}
\fancyfoot[C]{\itemCourse}
\fancyfoot[R]{Página \thepage}

% para el codigo fuente
\usepackage{listings}
\usepackage{color, colortbl}
\definecolor{dkgreen}{rgb}{0,0.6,0}
\definecolor{gray}{rgb}{0.5,0.5,0.5}
\definecolor{mauve}{rgb}{0.58,0,0.82}
\definecolor{codebackground}{rgb}{0.95, 0.95, 0.92}
\definecolor{tablebackground}{rgb}{0.8, 0, 0}

\lstset{frame=tb,
	language=bash,
	aboveskip=3mm,
	belowskip=3mm,
	showstringspaces=false,
	columns=flexible,
	basicstyle={\small\ttfamily},
	numbers=none,
	numberstyle=\tiny\color{gray},
	keywordstyle=\color{blue},
	commentstyle=\color{dkgreen},
	stringstyle=\color{mauve},
	breaklines=true,
	breakatwhitespace=true,
	tabsize=3,
	backgroundcolor= \color{codebackground},
}
\usepackage{color}
    \definecolor{bat-bg}{rgb}{.16,.19,.2}
    \definecolor{bat-norm}{rgb}{.9,.9,.9}
    \definecolor{bat-comm}{rgb}{.4,.45,.48}
    \definecolor{bat-kw}{rgb}{.57,.78,.39}
    \definecolor{bat-str}{rgb}{.2,.38,.62}
    \definecolor{arch-celeste}{rgb}{0.263, 0.765, 0.961}
    \definecolor{arch-lime}{rgb}{0.588, 0.773, 0.506}



\usepackage{listings}

\lstset{
    language={[WinXP]command.com},
    basicstyle=\color{bat-norm}\footnotesize\ttfamily,
    numbers=left,
    numberstyle=\normalsize,
    numbersep=7pt,
    backgroundcolor=\color{bat-bg},  
    commentstyle=\color{bat-comm},
    keywordstyle=\color{arch-celeste},
    numberstyle=\tiny\color{bat-bg},
    stringstyle=\color{arch-lime},
    tabsize=2
}

\begin{document}
	
	\vspace*{10pt}
	
	\begin{center}	
		\fontsize{17}{17} \textbf{ Informe de Laboratorio \itemPracticeNumber}
	\end{center}
	\centerline{\textbf{\Large Tema: \itemTheme}}
	%\vspace*{0.5cm}	

	\begin{flushright}
		\begin{tabular}{|M{2.5cm}|N|}
			\hline 
			\rowcolor{tablebackground}
			\color{white} \textbf{Nota}  \\
			\hline 
			     \\[30pt]
			\hline 			
		\end{tabular}
	\end{flushright}	

	\begin{table}[H]
		\begin{tabular}{|x{4.7cm}|x{4.8cm}|x{4.8cm}|}
			\hline 
			\rowcolor{tablebackground}
			\color{white} \textbf{Estudiante} & \color{white}\textbf{Escuela}  & \color{white}\textbf{Asignatura}   \\
			\hline 
			{\itemStudent \par \itemEmail} & \itemSchool & {\itemCourse \par Semestre: \itemSemester \par Código: \itemCourseCode}     \\
			\hline 			
		\end{tabular}
	\end{table}		
	
	\begin{table}[H]
		\begin{tabular}{|x{4.7cm}|x{4.8cm}|x{4.8cm}|}
			\hline 
			\rowcolor{tablebackground}
			\color{white}\textbf{Laboratorio} & \color{white}\textbf{Tema}  & \color{white}\textbf{Duración}   \\
			\hline 
			\itemPracticeNumber & \itemTheme & 04 horas   \\
			\hline 
		\end{tabular}
	\end{table}
	
	\begin{table}[H]
		\begin{tabular}{|x{4.7cm}|x{4.8cm}|x{4.8cm}|}
			\hline 
			\rowcolor{tablebackground}
			\color{white}\textbf{Semestre académico} & \color{white}\textbf{Fecha de inicio}  & \color{white}\textbf{Fecha de entrega}   \\
			\hline 
			\itemAcademic & \itemInput &  \itemOutput  \\
			\hline 
		\end{tabular}
	\end{table}
	
	\section{Tarea}
	\begin{itemize}		
		\item Analice, complete y pruebe el Código de la clase DemoBatalla
            \item Solucionar la Actividad 4 de la Práctica 1 pero usando arreglo de objetos
            \item Solucionar la Actividad 5 de la Práctica 1 pero usando arreglo de objetos

	\end{itemize}
		
	\section{Equipos, materiales y temas utilizados}
	\begin{itemize}
		\item Sistema Operativo ArchCraft GNU Linux 64 bits Kernell
		\item NeoVim
		\item OpenJDK 64-Bit 20.0.1 
		\item Git 2.42.0
		\item Cuenta en GitHub con el correo institucional.
		\item Programación Orientada a Objetos.
		\item Creacion de programas con CLI	
	\end{itemize}
	
	\section{URL de Repositorio Github}
	\begin{itemize}
		\item URL del Repositorio GitHub para clonar o recuperar.
		\item \url{https://github.com/JhonatanDczel/fp2-23b.git}
		\item URL para el laboratorio 02 en el Repositorio GitHub.
		\item \url{https://github.com/JhonatanDczel/fp2-23b/tree/main/fase01/lab03}
	\end{itemize}


	\section{Actividad 1: Completar el codigo de DemoBatalla}
	
	\subsection{Implementacion del metodo Mostrar Naves}
	\begin{itemize}	
		\item Necesitamos un metodo que muestre los datos de una flota
		\item Para eso debera recibir un arreglo de naves
	\end{itemize}	
		
	\begin{lstlisting}[language=java,caption={Metodo mostrarNaves}][H]
  public static void mostrarNaves(Nave [] flota){
    for(int i = 0; i < flota.length; i++){
      System.out.println("Nave numero " + i + ":");
      System.out.println(flota[i].getNombre());
    }
  }
	\end{lstlisting}
\begin{itemize}
    \item Como vemos, el metodo recorre el arreglo y va imprimiendo el nombre de cada nave que encuentra en la posicion i
    \item para evitar un error de desbordamiento de limites se usa flota.length
    \item Ahora haremos una optimizacion usando el ciclo for each
\end{itemize}
\begin{lstlisting}[languaje=java, caption={Optimizacion del metodo mostrarNaves}]
  public static void mostrarNaves(Nave [] flota){
    System.out.println("Mostrando las naves creadas: ");
    for(Nave n : flota){
      System.out.println(n.getNombre());
    } 
  }    
\end{lstlisting}
\begin{itemize}
    \item Esta version es mas eficiente ya que no es necesario conocer la ubicacion del indice
\end{itemize}

    \subsection{Ejecucion}
    \begin{lstlisting}[language=bash, caption={Ejecucion del codigo}]
    Naves creadas:
    Nave numero 1:
    Ninha
    Nave numero 2:
    Pinta
    Nave numero 3:
    SantaMaria

    \end{lstlisting}

    \begin{itemize}
        \item Asi es como se ve la ejecucion en consola, con un ejemplo generico
    \end{itemize}

    
        \subsection{Metodo mostrar naves por nombre}
        \begin{lstlisting}[language=java, caption={Metodo mostrarPorNombre}]
    public static void mostrarPorNombre(Nave [] flota, String nombre){
        System.out.println();
        int i = 1;
        for(Nave n : flota){
          if(n.getNombre().equals(nombre)){
            System.out.println("Nave " + i + ":");
            mostrarNave(n);
            i++;
          }
        }
        if(i == 1)
          System.out.println("No se han encontrado naves con ese nombre");
      }
        \end{lstlisting}
        \begin{itemize}
            \item Como podemos ver, el metodo recorre el arreglo que se le da en busca de naves con el mismo nombre
            \item Para trabajar sin distinciones entre mayusculas y minusculas implementaremos el metodo toLowerCase, para trabajar estandarizadamente y evitar conflictos:
        \end{itemize}
        \begin{lstlisting}[language=java, caption={Metodo mostrarPorNombre}]
  public static void mostrarPorNombre(Nave [] flota, String nombre){
    nombre = nombre.toLowerCase();
    System.out.println();
    int i = 1;
    for(Nave n : flota){
      if(n.getNombre().toLowerCase().equals(nombre)){
        System.out.println("Nave " + i + ":");
        mostrarNave(n);
        i++;
      }
    }
    if(i == 1)
      System.out.println("No se han encontrado naves con ese nombre");

  }
        \end{lstlisting}

        \subsection{Metodo Mostrar por puntos}
        \begin{itemize}
            \item Ahora necesitamos un metodo que muestre las naves con cantidad de puntos igual o menor al que el usuario entre por teclado
        \end{itemize}
        \begin{lstlisting}[language=java, caption={Implementacion del metodo mostrar por puntos}]
  public static void mostrarPorPuntos(Nave [] flota, int pts){
    System.out.println();
    int i = 1;
    for(Nave n : flota){
      if(n.getPuntos() <= pts){
        System.out.println("Nave " + i + ":");
        mostrarNave(n);
        System.out.println();
        i++;
      }
    }  
  }
        \end{lstlisting}
        \begin{itemize}
            \item Recorremos el arreglo comprobando si los puntos son iguales o menores al que se ingreso, en caso serlo se imprimen los datos de la nave
            \item Ahora necesitamos un caso "default" por si no se encuentran naves
        \end{itemize}
        \begin{lstlisting}[caption={Condicion de default}]
    if(i == 1){
      System.out.println("No se han encontrado naves");
    } 
        \end{lstlisting}

        \subsection{Metodo contarLetrasRestantes}
        \begin{itemize}
            \item Ahora necesitamos un metodo para mostrar la nave con mas puntos
            \item Para eso vamos a recorrer el ciclo y hacer una simple comprobacion con un dato pivote
        \end{itemize}
            \begin{lstlisting}[language=java, caption={Metodo Mostrar Mayor Puntos}]
  public static Nave mostrarMayorPuntos(Nave [] flota){
    Nave mayor = flota[0];
    for(int i = 0; i < flota.length; i++){
      if(flota[i].getPuntos() > mayor.getPuntos())
        mayor = flota[i];
    }
    return mayor;
  }
            \end{lstlisting}
        \begin{itemize}
            \item Recorremos el arreglo en busca del mayor puntaje
            \item Al final, el metodo devuelve la nave con el mayor puntaje, usando el indice que nos sirvio como indicador del mayor puntaje en cada vuelta
            
        \end{itemize}
        
        \subsection{Metodo para desordenar un array de Naves}
        \begin{itemize}
            \item Ahora necesitamos desordenar los elementos de un array de naves dado
            \item Para eso usaremos la clase Random para generar numeros aleatorios
            \item La estructura constara de un ciclo for que itere sobre cada uno de los elementos y un ciclo while dentro que se asegure de encontrar lugar para cada uno:
        \end{itemize}
        \begin{lstlisting}[language=java, caption={Ciclo while}]
      boolean ubicado = false;
      while(!ubicado){
        int numRandom = random.nextInt(flota.length);
        if(nuevaFlota[numRandom] == null){
          nuevaFlota[numRandom] = flota[i];
          ubicado = true;
          System.out.println("Nave " + i + " ahora ubicada en: " + numRandom);
        }
      }
        \end{lstlisting}
        \begin{itemize}
            \item El ciclo while solo terminara cuando se haya encontrado un lugar vacio para el elemento actual
            \item El numero random se calcula en funcion a los indices del arreglo, este numca podra ser igual a la longitud para evitar errores de desbordamiento de limite
        \end{itemize}
        \begin{lstlisting}[language=java, caption={Metodo completo}]
  public static Nave[] desordenar(Nave[] flota){
    Random random = new Random();
    Nave[] nuevaFlota = new Nave[flota.length];
    
    for(int i = 0; i < flota.length; i++){
      boolean ubicado = false;
      while(!ubicado){
        int numRandom = random.nextInt(flota.length);
        if(nuevaFlota[numRandom] == null){
          nuevaFlota[numRandom] = flota[i];
          ubicado = true;
          System.out.println("Nave " + i + " ahora ubicada en: " + numRandom);
        }
      }
    }
    return nuevaFlota;

  }
        \end{lstlisting}
        \begin{itemize}
            \item Ahora agregamos la estructura for para asegurarnos de recorrer todos los elementos, y finalmente retornamos el arreglo de Naves
        \end{itemize}
        
        \subsection{Commits importantes}
        \begin{itemize}
            \item Este es un registro de los commits mas importantes
            \item Estan extraidos del registro de commits al hacer git log
        \end{itemize}
        \begin{lstlisting}[language=bash, caption={Commits}]
    commit 689ce4e09e886234c9a67bf2205bf00e943f6e16
    Author: JhonatanDczel <jariasq@unsa.edu.pe>
    Date:   Sun Sep 24 20:40:02 2023 -0500
    
        Actividad 1: Implementacion del metodo Mostrar Naves
    
    commit 4a2c4c69527f7d5ce9194f4a28ae5eb8d874424d
    Author: JhonatanDczel <jariasq@unsa.edu.pe>
    Date:   Sun Sep 24 23:25:02 2023 -0500
    
        Actividad 1: Implementacion del metodo Mostrar Naves Por Nombre
    
    commit 42a4a4e1302014164874c6b1f84c058192d0870a
    Author: JhonatanDczel <jariasq@unsa.edu.pe>
    Date:   Sun Sep 24 23:40:26 2023 -0500
    
        Actividad 1: Implementacion del metodo Mostrar por Puntos
    
    commit 314a4ec7803f41d98d669fa97931009c74d6f6cb
    Author: JhonatanDczel <jariasq@unsa.edu.pe>
    Date:   Mon Sep 25 00:39:00 2023 -0500
    
        Actividad 1: Implementacion del metodo para mostrar nave con mayor puntaje
    
    commit bc49db90fdedaade7c05d8c526608e093f0de518
    Author: JhonatanDczel <jariasq@unsa.edu.pe>
    Date:   Mon Sep 25 10:38:18 2023 -0500
    
        Actividad 1: Terminando el metodo para desordenar un array de objetos aleatoriamente
    

        \end{lstlisting}
        \begin{itemize}
            \item Cada commit representa un metodo implementado
            \item Adicionalmente cada metodo esta acompanhado de su optimizacion
        \end{itemize}
    \subsection{Ejecucion}
    \begin{itemize}
        \item Tendremos ahora una demostracion sobre el DemoBatalla:
    \end{itemize}

        \begin{lstlisting}[language=bash, caption={Ejecucion en la linea de comandos}]
    Naves creadas:
    Nave 1:
    Nombre: Independencia
    Estado: true
    Puntos: 65
    
    Nave 2:
    Nombre: Huascaran
    Estado: false
    Puntos: 65
    
    Nave 3:
    Nombre: Union
    Estado: false
    Puntos: 100
    
    Desordenando las naves:
    Nave 0 ahora ubicada en: 1
    Nave 1 ahora ubicada en: 2
    Nave 2 ahora ubicada en: 0
    
    Mostrar naves por nombre, ingrese un nombre:
    Union
    
    Nave 1:
    Nombre: Union
    Estado: false
    Puntos: 100
    
    Mostrar naves por puntos, ingrese una cantidad:
    35
    
    No se han encontrado naves
    
    Nave con mayor número de puntos: Nave@5e2de80c
        \end{lstlisting}
        \begin{itemize}
            \item Como hemos podido ver, todos los metodos funcionan correctamente
        \end{itemize}
\section{Actividad 2: Rehacer la actividad 4 del laboratorio 01, pero con array de Objetos}
    \begin{itemize}
        \item En aquel codigo, todo se maneja en variables simples
        \item Unicamente necesitamos crer un array tal que asi Soldado[], y reemplazar las variables por metodos de Soldado
    \end{itemize}
        \begin{lstlisting}[language=java, caption={Array de Objetos}]
    Soldado[] ejercito = new Soldado[5];

    for(int i = 0; i < 5; i++){
      System.out.println("\nIngrese el nombre del soldado numero " + (i + 1) + ":");
      ejercito[i] = new Soldado(sc.next());

      System.out.println("Ingrese el nivel de vida del soldado numero " + (i + 1) + ":");
      ejercito[i].setLife(sc.nextInt());
    }
        \end{lstlisting}
\begin{itemize}
    \item Ahora, para mostrar los datos, mas de lo mismo, solo tenemos que cambiar variables por metodos
\end{itemize}
        \begin{lstlisting}[language=java, caption={Array de Objetos}]
    System.out.println("\n=====DATOS DE SOLDADOS=====");
    System.out.println("Soldado: " + ejercito[0].getName() + " \nNivel de vida: " + ejercito[0].getLife() + "\n");
    System.out.println("Soldado: " + ejercito[1].getName() + " \nNivel de vida: " + ejercito[1].getLife() + "\n");
    System.out.println("Soldado: " + ejercito[2].getName() + " \nNivel de vida: " + ejercito[2].getLife() + "\n");
    System.out.println("Soldado: " + ejercito[3].getName() + " \nNivel de vida: " + ejercito[3].getLife() + "\n");
    System.out.println("Soldado: " + ejercito[4].getName() + " \nNivel de vida: " + ejercito[4].getLife() + "\n");
        \end{lstlisting}
\subsection{Ejecucion}
\begin{itemize}
    \item A continuacion veremos un ejemplo de ejecucion:
\end{itemize}
\begin{lstlisting}
    Ingrese el nombre del soldado numero 1:
    Punchinelo
    Ingrese el nivel de vida del soldado numero 1:
    12
    
    Ingrese el nombre del soldado numero 2:
    Roger
    Ingrese el nivel de vida del soldado numero 2:
    54
    
    Ingrese el nombre del soldado numero 3:
    Klibre
    Ingrese el nivel de vida del soldado numero 3:
    65
    
    =====DATOS DE SOLDADOS=====
    Soldado: Punchinelo
    Nivel de vida: 12
    
    Soldado: Roger
    Nivel de vida: 54
    
    Soldado: Klibre
    Nivel de vida: 65

\end{lstlisting}
\section{Actividad 3: Rehacer la actividad 05 del laboratorio 01, pero con array de Objetos}
\begin{itemize}
    \item Ahora es mas simple aun, dado que en ese laboratorio trabajamos con Arreglo de Strings:
\end{itemize}
\begin{lstlisting}
    String[] army1 = initializeArmy(); 
    String[] army2 = initializeArmy();

    System.out.println("╔════════════════════════════╗");
    System.out.println("║    Welcome to the Battle   ║");
    System.out.println("║       Simulator Game!      ║");
    System.out.println("╚════════════════════════════╝");
    
    System.out.println("\n***** Prepare for battle! *****");

    displayArmy(army1);
    displayArmy(army2);

    System.out.println(whoWins(army1, army2));
\end{lstlisting}
\begin{itemize}
    \item Deberemos sustiruir los String[] por Soldado[]
    \item Es un trabajo trivial que no cambia en nada la estructura del codigo.. pero aqui esta
\end{itemize}
\begin{lstlisting}
  public static void main(String[] args){
    Soldado[] army1 = initializeArmy(); 
    Soldado[] army2 = initializeArmy();

    System.out.println("╔════════════════════════════╗");
    System.out.println("║    Welcome to the Battle   ║");
    System.out.println("║       Simulator Game!      ║");
    System.out.println("╚════════════════════════════╝");
    
    System.out.println("\n***** Prepare for battle! *****");

    displayArmy(army1);
    displayArmy(army2);

    System.out.println(whoWins(army1, army2));

  }
\end{lstlisting}
\begin{itemize}
    \item Adicionalmente tambien tenemos que modificar los metodos que usa este codigo
\end{itemize}
\subsection{Ejecucion}
\begin{itemize}
    \item La ejecucion es exactamente la misma:
\end{itemize}
\begin{lstlisting}
    Ingrese el nombre del soldado numero 1: Ricardo
    Ingrese el nivel de vida del soldado numero 1: 54
    
    Ingrese el nombre del soldado numero 2: Potter
    Ingrese el nivel de vida del soldado numero 2: 65
    
    Ingrese el nombre del soldado numero 3: Hilario
    Ingrese el nivel de vida del soldado numero 3: 65
    
    =====DATOS DE SOLDADOS=====
    Soldado: Ricardo
    Nivel de vida: 54
    
    Soldado: Potter
    Nivel de vida: 65
    
    Soldado: Hilario
    Nivel de vida: 65
\end{lstlisting}
\section{Commits importantes (Actividad 2 y 3)}
\begin{itemize}
    \item Dada la naturaleza trivial del trabajo, todo se hizo en 2 commits
\end{itemize}
\begin{lstlisting}
    commit ac2abb32f6f077e13f0202653e456ccbeacfee2e (HEAD -> main, origin/main)
    Author: JhonatanDczel <jariasq@unsa.edu.pe>
    Date:   Mon Sep 25 10:58:13 2023 -0500
    
        Actividad 3: hacer la actividad 5 del laboratorio 1 pero con arreglo de objetos
    
    commit ccd2d7c9d210b83aa4903a69d0eb6879a173ce52
    Author: JhonatanDczel <jariasq@unsa.edu.pe>
    Date:   Mon Sep 25 10:52:43 2023 -0500
    
        Actividad 2: hacer La actividad 4 del laboratorio 1 con arreglo de objetos
\end{lstlisting}
        
	\section{\textcolor{red}{Rúbricas}}
	
	\subsection{\textcolor{red}{Entregable Informe}}
	\begin{table}[H]
		\caption{Tipo de Informe}
		\setlength{\tabcolsep}{0.5em} % for the horizontal padding
		{\renewcommand{\arraystretch}{1.5}% for the vertical padding
		\begin{tabular}{|p{3cm}|p{12cm}|}
			\hline
			\multicolumn{2}{|c|}{\textbf{\textcolor{red}{Informe}}}  \\
			\hline 
			\textbf{\textcolor{red}{Latex}} & \textcolor{blue}{El informe está en formato PDF desde Latex,  con un formato limpio (buena presentación) y facil de leer.}   \\ 
			\hline 
			
			
		\end{tabular}
	}
	\end{table}
	
	\clearpage
	
	\subsection{\textcolor{red}{Rúbrica para el contenido del Informe y demostración}}
	\begin{itemize}			
		\item El alumno debe marcar o dejar en blanco en celdas de la columna \textbf{Checklist} si cumplio con el ítem correspondiente.
		\item El alumno debe autocalificarse en la columna \textbf{Estudiante} de acuerdo a la siguiente tabla:
	
		\begin{table}[ht]
			\caption{Niveles de desempeño}
			\begin{center}
			\begin{tabular}{ccccc}
    			\hline
    			 & \multicolumn{4}{c}{Nivel}\\
    			\cline{1-5}
    			\textbf{Puntos} & Insatisfactorio 25\%& En Proceso 50\% & Satisfactorio 75\% & Sobresaliente 100\%\\
    			\textbf{2.0}&0.5&1.0&1.5&2.0\\
    			\textbf{4.0}&1.0&2.0&3.0&4.0\\
    		\hline
			\end{tabular}
		\end{center}
	\end{table}	
	
	\end{itemize}
	
	\begin{table}[H]
		\caption{Rúbrica para contenido del Informe y demostración}
		\setlength{\tabcolsep}{0.5em} % for the horizontal padding
		{\renewcommand{\arraystretch}{1.5}% for the vertical padding
		%\begin{center}
		\begin{tabular}{|p{2.7cm}|p{7cm}|x{1.3cm}|p{1.2cm}|p{1.5cm}|p{1.1cm}|}
			\hline
    		\multicolumn{2}{|c|}{Contenido y demostración} & Puntos & Checklist & Estudiante & Profesor\\
			\hline
			\textbf{1. GitHub} & Hay enlace URL activo del directorio para el  laboratorio hacia su repositorio GitHub con código fuente terminado y fácil de revisar. &2 &X &2 & \\ 
			\hline
			\textbf{2. Commits} &  Hay capturas de pantalla de los commits más importantes con sus explicaciones detalladas. (El profesor puede preguntar para refrendar calificación). &4 &X &4 & \\ 
			\hline 
			\textbf{3. Código fuente} &  Hay porciones de código fuente importantes con numeración y explicaciones detalladas de sus funciones. &2 &X &2 & \\ 
			\hline 
			\textbf{4. Ejecución} & Se incluyen ejecuciones/pruebas del código fuente  explicadas gradualmente. &2 &X &2 & \\ 
			\hline			
			\textbf{5. Pregunta} & Se responde con completitud a la pregunta formulada en la tarea.  (El profesor puede preguntar para refrendar calificación).  &2 &X &1 & \\ 
			\hline	
			\textbf{6. Fechas} & Las fechas de modificación del código fuente estan dentro de los plazos de fecha de entrega establecidos. &2 &X &2 & \\ 
			\hline 
			\textbf{7. Ortografía} & El documento no muestra errores ortográficos. &2 &X &1 & \\ 
			\hline 
			\textbf{8. Madurez} & El Informe muestra de manera general una evolución de la madurez del código fuente,  explicaciones puntuales pero precisas y un acabado impecable.   (El profesor puede preguntar para refrendar calificación).  &4 &X &3 & \\ 
			\hline
			\multicolumn{2}{|c|}{\textbf{Total}} &20 & &17 & \\ 
			\hline
		\end{tabular}
		%\end{center}
		%\label{tab:multicol}
		}
	\end{table}
	
\clearpage
	
%\clearpage
%\bibliographystyle{apalike}
%\bibliographystyle{IEEEtranN}
%\bibliography{bibliography}
			
\end{document}
