%USEPACKAGES
\input{head/pkgs}

%Lab data
\newcommand{\itemEmail}{jariasq@unsa.edu.pe}
\newcommand{\itemStudent}{Jhonatan David Arias Quispe}
\newcommand{\itemCourse}{Fundamentos de Programacion 2}
\newcommand{\itemCourseCode}{1701213}
\newcommand{\itemSemester}{II}
\newcommand{\itemUniversity}{Universidad Nacional de San Agustín de Arequipa}
\newcommand{\itemFaculty}{Facultad de Ingeniería de Producción y Servicios}
\newcommand{\itemDepartment}{Departamento Académico de Ingeniería de Sistemas e Informática}
\newcommand{\itemSchool}{Escuela Profesional de Ingeniería de Sistemas}
\newcommand{\itemAcademic}{2023 - B}
\newcommand{\itemInput}{Del 12 Octubre 2023}
\newcommand{\itemOutput}{Al 15 Octubre 2023}
\newcommand{\itemPracticeNumber}{06}
\newcommand{\itemTheme}{Array List}
\newcommand{\itemUrl}{\url{https://github.com/JhonatanDczel/fp2-23b/tree/main/fase02/lab\itemPracticeNumber}}


%Format Document
\input{head/format}

% Code Format
\input{head/codeFormat}

\begin{document}
%Data table

	
	\vspace*{10pt}
	
	\begin{center}	
		\fontsize{17}{17} \textbf{ Informe de Laboratorio \itemPracticeNumber}
	\end{center}
	\centerline{\textbf{\Large Tema: \itemTheme}}
	\vspace*{0.5cm}	

	\begin{flushright}
		\begin{tabular}{|M{2.5cm}|N|}
			\hline 
			\rowcolor{tablebackground}
			\color{white} \textbf{Nota}  \\
			\hline 
			     \\[30pt]
			\hline 			
		\end{tabular}
	\end{flushright}	

	\begin{table}[H]
		\begin{tabular}{|x{4.9cm}|x{4.3cm}|x{5.1cm}|}
			\hline 
			\rowcolor{tablebackground}
			\color{white} \textbf{Estudiante} & \color{white}\textbf{Escuela}  & \color{white}\textbf{Asignatura}   \\
			\hline 
			{\itemStudent \par \itemEmail} & \itemSchool & {\itemCourse \par Semestre: \itemSemester \par Código: \itemCourseCode}     \\
			\hline 			
		\end{tabular}
	\end{table}		
	
	\begin{table}[H]
		\begin{tabular}{|x{4.7cm}|x{4.8cm}|x{4.8cm}|}
			\hline 
			\rowcolor{tablebackground}
			\color{white}\textbf{Laboratorio} & \color{white}\textbf{Tema}  & \color{white}\textbf{Duración}   \\
			\hline 
			\itemPracticeNumber & \itemTheme & 04 horas   \\
			\hline 
		\end{tabular}
	\end{table}
	
	\begin{table}[H]
		\begin{tabular}{|x{4.7cm}|x{4.8cm}|x{4.8cm}|}
			\hline 
			\rowcolor{tablebackground}
			\color{white}\textbf{Semestre académico} & \color{white}\textbf{Fecha de inicio}  & \color{white}\textbf{Fecha de entrega}   \\
			\hline 
			\itemAcademic & \itemInput &  \itemOutput  \\
			\hline 
		\end{tabular}
	\end{table}

	\section{Actividades}
	\begin{itemize}		
      \item Al ejecutar el videojuego, el programa deberá dar las opciones:
      \item 1. Juego rápido (tal cual como en el laboratorio 11)
Al acabar el juego mostrar las opciones de volver a jugar y de volver al
menú principal. También se deberá tener la posibilidad de cancelar el
juego actual en cualquier momento, permitiendo escoger entre empezar
un juego totalmente nuevo o salir al menú principal.
      \item 2. Juego personalizado: permite festionar ejércitos. Primero se generan los 2 ejércitos con sus respectivos soldados y se muestran sus datos. Luego se tendrá que escoger cuál de los 2 ejércitos se va a gestionar, después se mostrarán las siguientes opciones:
        \begin{itemize}
          \item Crear Soldado: permitirá crear un nuevo soldado personalizado
y añadir al final del ejército (recordar que límite es de 10
soldados por ejército)
          \item Eliminar Soldado (no debe permitir un ejército vacío)
          \item Clonar Soldado (crea una copia exacta del soldado) y se añade
al final del ejército (recordar que límite es de 10 soldados por
ejército)
          \item Modificar Soldado (con submenú para cambiar alguno de los
atributos nivelAtaque, nivelDefensa, vidaActual)
          \item Comparar Soldados (verifica si atributos: nombre, nivelAtaque,
nivelDefensa, vidaActual y vive son iguales)
          \item Intercambiar Soldados (intercambia 2 soldados en sus posiciones
en la estructura de datos del ejército)
          \item Ver soldado (Búsqueda por nombre)
          \item Ver ejército
          \item Sumar niveles (usando Method-Call Chaining), calcular las
sumatorias de nivelVida, nivelAtaque, nivelDefensa, velocidad de
todos los soldados de un ejército 
            \begin{itemize}
              \item Por ejemplo, si ejército tendría 3 soldados:
              \item s=s1.sumar(s2).sumar(s3);
              \item s es un objeto Soldado nuevo que contendría las
sumatorias de los 4 atributos indicados de los 3 soldados.
Ningún soldado cambia sus valores
            \end{itemize}
          \item Jugar (se empezará el juego con los cambios realizados) y con
las mismas opciones de la opción 1.
          \item Volver (muestra el menú principal)
Después de escoger alguna de las opciones 1) a 9) se podrá volver a
elegir uno de los ejércitos y se mostrarán las opciones 1) a 11)
        \end{itemize} 
        \item 3. Salir
	\end{itemize}
		
	\section{Equipos, materiales y temas utilizados}
	\begin{itemize}
		\item Sistema Operativo ArchCraft GNU Linux 64 bits Kernell
		\item NeoVim
		\item OpenJDK 64-Bit 20.0.1 
		\item Git 2.42.0
		\item Cuenta en GitHub con el correo institucional.
		\item Programación Orientada a Objetos.
		\item Creacion de programas con CLI	
            \item Bilioteca Graphics (origen propio)
	\end{itemize}
	\section{URL de Repositorio Github}
	\begin{itemize}
            \item URL del Repositorio GitHub para clonar o recuperar.
            \item \url{https://github.com/JhonatanDczel/fp2-23b.git}
            \item URL para el laboratorio \itemPracticeNumber{} en el Repositorio GitHub.
            \item \itemUrl
	\end{itemize}




\section{Actividad 1}
\subsection{Para las pruebas de ejecucion}
\begin{itemize}
    \item Para la prueba de ejecucion de cada uno de los metodos que desarrollaremos, vamos a usar las siguientes naves:
\end{itemize}
\begin{lstlisting}
    Naves creadas:
    
    Nave 1:
    Nombre: cesar
    Estado: true
    Puntos: 23
    
    Nave 2:
    Nombre: Alvaro
    Estado: true
    Puntos: 6
    
    Nave 3:
    Nombre: Bannia
    Estado: true
    Puntos: 89
\end{lstlisting}

\section{Completando los metodos faltantes}
\begin{itemize}
    \item La acitivdad del laboratorio 04 nos pide crear los siguientes metodos:
\end{itemize}
\subsection{Busqueda Lineal}
\begin{itemize}
    \item La busqueda lineal es una busqueda simple, en la que se recorre todo el arreglo en busca de una informacion dada
\end{itemize}
\begin{lstlisting}[language=java, caption={Metodo de busqueda lineal}]
      public static int busquedaLinealNombre(Nave[] flota, String s){
        for(int i = 0; i < flota.length; i++){
          if(flota[i].getNombre().equals(s))
            return i;
        }
        return -1;
      } 
\end{lstlisting}

\begin{itemize}
    \item El metodo devuelve la posicion del elemento en caso encontrarlo
    \item Y de lo contrario devuelve -1 lo que indica que el elemento no se encontro
\end{itemize}
\subsubsection{Ejecucion}
\begin{itemize}
    \item Ahora veremos la prueba del metodo:
\end{itemize}
    \begin{lstlisting}[language=bash, caption={Prueba del metodo}]
    Busqueda lineal, inserte un nombre:
    Bannia
    Nombre: Bannia
    Estado: true
    Puntos: 89
    
\end{lstlisting}

\subsection{Ordenamiento por nombre: burbuja}
\begin{itemize}
    \item Ahora veremos como implementar el bubble sort para ordenar las naves con respecto a sus nombres
\end{itemize}
    \begin{lstlisting}[language=java, caption={Metodo de ordenamiento burbuja por nombre}]
      public static void ordenarPorNombreBurbuja(Nave[] flota){
        for(int i = 0; i < flota.length - 1; i++){
          for(int j = 0; j < flota.length - 1 - i; j++){
            if(esMayor(flota[j].getNombre(), flota[j + 1].getNombre()))
              intercambiar(flota, j, j + 1);
          }
        }
      }
\end{lstlisting}
\begin{itemize}
    \item Como vemos, estamos utilizando un metodo que compara dos strings y devuelve cual es mayor lexicograficamente:
\end{itemize}
\begin{lstlisting}[language=java, caption={Sub metodo para comparar dos strings}]
  public static boolean esMayor(String s1, String s2){
    return s1.compareToIgnoreCase(s2) > 0;
  }
\end{lstlisting}
\begin{itemize}
    \item Ahora tenemos una manera de saber el orden de precedencia entre strings
    \item Adicionalmente a esto necesitamos otro metodo para intercambiar elementos de un arreglo de naves:
\end{itemize}
\begin{lstlisting}[language=java, caption={Sub metodo para intercambiar dos elementos}]
      public static void intercambiar(Nave[] flota, int i, int j){
        Nave aux = flota[i];
        flota[i] = flota [j];
        flota[j] = aux;
      }
\end{lstlisting}
\subsubsection{Ejecucion}
\begin{itemize}
    \item Ahora veremos la prueba del metodo:
\end{itemize}
    \begin{lstlisting}[language=bash, caption={Prueba del metodo}]
    Ahora usaremos el algoritmo de ordenamiento burbuja, con respecto a Nombres.
    Naves ordenadas:
    
    
    Nave 1:
    Nombre: Alvaro
    Estado: true
    Puntos: 6
    
    Nave 2:
    Nombre: Bannia
    Estado: true
    Puntos: 89
    
    Nave 3:
    Nombre: cesar
    Estado: true
    Puntos: 23
    
\end{lstlisting}
\subsection{Ordenamiento por puntos: burbuja}
\begin{itemize}
    \item Ahora veremos como ordenar los elementos de un arreglo respecto a sus puntos usando el algoritmo burbuja
    \item Hacer esto es muy similar al ordenamiento por nombre, unicamente cambiando getNombre por getPuntos:
\end{itemize}
\begin{lstlisting}[language=java, caption={Metodo para ordenar por puntos}]
      public static void ordenarPorPuntosBurbuja(Nave[] flota){
        for(int i = 0; i < flota.length - i; i++){
          for(int j = 0; j < flota.length - 1 - i; j++){
            if(flota[j].getPuntos() > flota[j + 1].getPuntos())
              intercambiar(flota, j, j + 1);
          }
        }
      }
\end{lstlisting}
\begin{itemize}
    \item Ahora en lugar de comparar dos strings, directamente se comparan los valores numericos de sus puntos
\end{itemize}
\subsubsection{Ejecucion}
\begin{itemize}
    \item Ahora veremos la prueba del metodo:
\end{itemize}
    \begin{lstlisting}[language=bash, caption={Prueba del metodo}]
    Ahora usaremos el algoritmo de ordenamiento burbuja, con respecto a Puntos.
    Naves ordenadas:
    
    
    Nave 1:
    Nombre: Alvaro
    Estado: true
    Puntos: 6
    
    Nave 2:
    Nombre: cesar
    Estado: true
    Puntos: 23
    
    Nave 3:
    Nombre: Bannia
    Estado: true
    Puntos: 89
     
\end{lstlisting}

\subsection{Ordenamiento por nombre: seleccion}
\begin{itemize}
    \item Ahora veremos el otro algoritmo de ordenamiento, el de seleccion
    \item La estructura basica del algoritmo, es ir seleccionando de izquierda a derecha los elementos mas chicos para ordenar el arreglo
    \item Su implementacion para ordenar una serie de naves lexicograficamente es la siguiente:
\end{itemize}
\begin{lstlisting}[language=java, caption={Metodo para ordenar por nombre}]
      public static void ordenarPorNombreSeleccion(Nave[] flota){
        for(int i = 0; i < flota.length - 1; i++){
          int min = i;
          for(int j = i + 1; j < flota.length; j++){
            if(esMayor(flota[min].getNombre(), flota[j].getNombre()))
              min = j;
          }
          intercambiar(flota, i, min);
        }
      }
\end{lstlisting}
\subsubsection{Ejecucion}
\begin{itemize}
    \item Ahora veremos la prueba del metodo:
\end{itemize}
    \begin{lstlisting}[language=bash, caption={Prueba del metodo}]
    Ahora usaremos el algoritmo de ordenamiento seleccion, con respecto a Nombres.
    Naves ordenadas:
    
    
    Nave 1:
    Nombre: Alvaro
    Estado: true
    Puntos: 6
    
    Nave 2:
    Nombre: Bannia
    Estado: true
    Puntos: 89
    
    Nave 3:
    Nombre: cesar
    Estado: true
    Puntos: 23
    
\end{lstlisting}


\subsection{Ordenamiento por puntos: seleccion}
\begin{itemize}
    \item Ahora usaremos este mismo algoritmo para ordenar el array con respecto a los puntos
    \item Es la misma estructura y el mismo sentido, en lugar de unsar getNombre, usamos getPuntos
\end{itemize}
\begin{lstlisting}[language=java, caption={Metodo para ordenar por puntos}]
      public static void ordenarPorPuntosSeleccion(Nave[] flota){
    for(int i = 0; i < flota.length - 1; i++){
      int min = i;
      for(int j = i + 1; j < flota.length; j++){
        if(flota[j].getPuntos() < flota[min].getPuntos())
          min = j;
      }
      intercambiar(flota, i, min);
    }
  }
\end{lstlisting}
\subsubsection{Ejecucion}
\begin{itemize}
    \item Ahora veremos la prueba del metodo:
\end{itemize}
    \begin{lstlisting}[language=bash, caption={Prueba del metodo}]
    Ahora usaremos el algoritmo de ordenamiento seleccion, con respecto a Puntos.
    Naves ordenadas:
    
    
    Nave 1:
    Nombre: Alvaro
    Estado: true
    Puntos: 6
    
    Nave 2:
    Nombre: cesar
    Estado: true
    Puntos: 23
    
    Nave 3:
    Nombre: Bannia
    Estado: true
    Puntos: 89
      
\end{lstlisting}

\subsection{Ordenamiento por nombre: insercion}
\begin{itemize}
    \item Ahora veremos el algoritmo de insercion
    \item La idea cae en como ordenamos nuestras cartas cuando tenemos una baraja, agarramos la carta y la "insertamos" en su lugar
    \item Vamos a hacer esto pero con codigo, empezamos:
\end{itemize}
\begin{lstlisting}[language=java, caption={Metodo para ordenar por nombre}]
      public static void ordenarPorNombreInsercion(Nave[] flota){
    for(int i = 1; i < flota.length; i ++){
      Nave actual = flota[i];
      int j = i - 1;
      while(j >= 0 && esMayor(flota[j].getNombre(), actual.getNombre())){
        flota[j + 1] = flota[j];
        j--;
      }
      flota[j + 1] = actual;
    }
  }
\end{lstlisting}
\begin{itemize}
    \item Estamos usando una variable para almacenar nuestro elemento a ubicar
    \item Luego recorremos tantos elementos a la derecha como elementos menores a nuestro pivote hayan
    \item Y finalmente colocamos nuestro pivote en la posicion que queda libre
\end{itemize}
\subsubsection{Ejecucion}
\begin{itemize}
    \item Ahora veremos la prueba del metodo:
\end{itemize}
    \begin{lstlisting}[language=bash, caption={Prueba del metodo}]
    Ahora usaremos el algoritmo de ordenamiento seleccion, con respecto a Nombres.
    Naves ordenadas:
    
    
    Nave 1:
    Nombre: Alvaro
    Estado: true
    Puntos: 6
    
    Nave 2:
    Nombre: Bannia
    Estado: true
    Puntos: 89
    
    Nave 3:
    Nombre: cesar
    Estado: true
    Puntos: 23
    
      
\end{lstlisting}





\subsection{Ordenamiento por puntos: insercion}
\begin{itemize}
    \item Ahora usaremos este mismo algoritmo para ordenar el array con respecto a los puntos
    \item Es la misma estructura y el mismo sentido, en lugar de unsar getNombre, usamos getPuntos
\end{itemize}
\begin{lstlisting}[language=java, caption={Metodo para ordenar por puntos}]
      public static void ordenarPorPuntosInsercion(Nave[] flota){
    for(int i = 1; i < flota.length; i++){
      Nave actual = flota[i];
      int j = i - 1;
      while(j >= 0 && flota[j].getPuntos() > actual.getPuntos()){
        flota[j + 1] = flota[j];
        j--;
      }
      flota[j + 1] = actual;
    }
  }
\end{lstlisting}
\subsubsection{Ejecucion}
\begin{itemize}
    \item Ahora veremos la prueba del metodo:
\end{itemize}
    \begin{lstlisting}[language=bash, caption={Prueba del metodo}]
    Ahora usaremos el algoritmo de ordenamiento insercion, con respecto a Puntos.
    Naves ordenadas:
    
    
    Nave 1:
    Nombre: Alvaro
    Estado: true
    Puntos: 6
    
    Nave 2:
    Nombre: cesar
    Estado: true
    Puntos: 23
    
    Nave 3:
    Nombre: Bannia
    Estado: true
    Puntos: 89      
\end{lstlisting}



\subsection{Busqueda binaria}
\begin{itemize}
    \item La busqueda binaria es una forma mas eficiente de buscar un elemento en una lista
    \item Como pre requisito, es que nuestro arreglo tiene que estar ordenado alfabeticamente, para eso usamos cualquier algoritmo de ordenamiento por nombre visto anteriormente
    \item El tiempo de complejidad es o(log n), lo que es muy bueno
\end{itemize}
\begin{lstlisting}[language=java, caption={Metodo para hacer una busqueda binaria}]
      public static int busquedaBinariaNombre(Nave[] flota, String s){
    ordenarPorNombreSeleccion(flota);
    int min = 0;
    int max = flota.length - 1;
    while(min <= max){
      int medio = (max + min) / 2;
      if(flota[medio].getNombre().equalsIgnoreCase(s))
        return medio;
      if(esMayor(s, flota[medio].getNombre()))
        min = medio + 1;
      else
        max = medio - 1;
    }
    return - 1;
  }
\end{lstlisting}
\subsubsection{Ejecucion}
\begin{itemize}
    \item Ahora veremos la prueba del metodo:
\end{itemize}
    \begin{lstlisting}[language=bash, caption={Prueba del metodo}]
    Busqueda binaria, inserte un nombre:
    cesar
    Nombre: cesar
    Estado: true
    Puntos: 23
\end{lstlisting}



\section{Commits importantes}
\begin{itemize}
    \item Ahora veremos el registro de los commits principales:
\end{itemize}
\begin{lstlisting}

commit 9b1f5460554796ac356fd9dcae4fc20bb3d3a123
Author: JhonatanDczel <jariasq@unsa.edu.pe>
Date:   Wed Sep 27 01:46:26 2023 -0500

    Actividad 1: Implementacion del metodo de ordenamiento de Puntos por insercion

commit a68dcb2c45cf068ed3fdf3733bfd3113f16abe6c
Author: JhonatanDczel <jariasq@unsa.edu.pe>
Date:   Wed Sep 27 01:18:23 2023 -0500

    Actividad 1: Implementacion del algoritmo de ordenamiento de puntos por seleccion

commit 18c3176724a4529cf0e77038215cdf5e180088ae
Author: JhonatanDczel <jariasq@unsa.edu.pe>
Date:   Wed Sep 27 01:07:19 2023 -0500

    Actividad 1: Implementacion del metodo de busqueda binaria por nombre

commit 0db1011f08fff2f98effb4123c5d2136ea68879d
Author: JhonatanDczel <jariasq@unsa.edu.pe>
Date:   Tue Sep 26 23:54:17 2023 -0500

    Actividad 1: Implementacion del metodo de ordenamiento por puntos usando el algoritmo burbuja

\end{lstlisting}
\begin{itemize}
    \item Cada commit representa el uso de un algoritmo diferente
    \item Esta el algoritmo burbuja, insercion y seleccion, asi como la busqueda binaria
\end{itemize}
%Rubricas de evaluacion
\include{foot/rubricas}	

\clearpage
	
%\clearpage
%\bibliographystyle{apalike}
%\bibliographystyle{IEEEtranN}
%\bibliography{bibliography}
			
\end{document}
