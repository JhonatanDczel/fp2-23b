
	
	\vspace*{10pt}
	
	\begin{center}	
		\fontsize{17}{17} \textbf{ Informe de Laboratorio \itemPracticeNumber}
	\end{center}
	\centerline{\textbf{\Large Tema: \itemTheme}}
	\vspace*{0.5cm}	

	\begin{flushright}
		\begin{tabular}{|M{2.5cm}|N|}
			\hline 
			\rowcolor{tablebackground}
			\color{white} \textbf{Nota}  \\
			\hline 
			     \\[30pt]
			\hline 			
		\end{tabular}
	\end{flushright}	

	\begin{table}[H]
		\begin{tabular}{|x{4.9cm}|x{4.3cm}|x{5.1cm}|}
			\hline 
			\rowcolor{tablebackground}
			\color{white} \textbf{Estudiante} & \color{white}\textbf{Escuela}  & \color{white}\textbf{Asignatura}   \\
			\hline 
			{\itemStudent \par \itemEmail} & \itemSchool & {\itemCourse \par Semestre: \itemSemester \par Código: \itemCourseCode}     \\
			\hline 			
		\end{tabular}
	\end{table}		
	
	\begin{table}[H]
		\begin{tabular}{|x{4.7cm}|x{4.8cm}|x{4.8cm}|}
			\hline 
			\rowcolor{tablebackground}
			\color{white}\textbf{Laboratorio} & \color{white}\textbf{Tema}  & \color{white}\textbf{Duración}   \\
			\hline 
			\itemPracticeNumber & \itemTheme & 04 horas   \\
			\hline 
		\end{tabular}
	\end{table}
	
	\begin{table}[H]
		\begin{tabular}{|x{4.7cm}|x{4.8cm}|x{4.8cm}|}
			\hline 
			\rowcolor{tablebackground}
			\color{white}\textbf{Semestre académico} & \color{white}\textbf{Fecha de inicio}  & \color{white}\textbf{Fecha de entrega}   \\
			\hline 
			\itemAcademic & \itemInput &  \itemOutput  \\
			\hline 
		\end{tabular}
	\end{table}

	\section{Actividades}
	\begin{itemize}		
      \item Cree un Proyecto llamado Laboratorio7
      \item Usted deberá crear las dos clases Soldado.java y VideoJuego4.java. Puede reutilizar lo desarrollado en Laboratorios anteriores.
      \item Del Soldado nos importa el nombre, puntos de vida, fila y columna (posición en el tablero).
      \item El juego se desarrollará en el mismo tablero de los laboratorios anteriores. Para el tablero utilizar la estructura de datos más adecuada.
      \item Tendrá 2 Ejércitos (utilizar la estructura de datos más adecuada). Inicializar el tablero con n soldados aleatorios entre 1 y 10 para cada Ejército. Cada soldado tendrá un nombre autogenerado: Soldado0X1, Soldado1X1, etc., un valor de puntos de vida autogenerado aleatoriamente [1..5], la fila y columna también autogenerados aleatoriamente (no puede haber 2 soldados en el mismo cuadrado). Se debe mostrar el tablero con todos los soldados creados y sus puntos de vida (usar caracteres como y otros y distinguir los de un ejército de los del otro ejército). Además de los datos del Soldado con mayor vida de cada ejército, el promedio de puntos de vida de todos los soldados creados por ejército, los datos de todos los soldados por ejército en el orden que fueron creados y un ranking de poder de todos los soldados creados por ejército (del que tiene más nivel de vida al que tiene menos) usando 2 diferentes algoritmos de ordenamiento. Finalmente, que muestre qué ejército ganará la batalla (indicar la métrica usada para decidir al ganador de la batalla). Hacer el programa iterativo.
	\end{itemize}
		
	\section{Equipos, materiales y temas utilizados}
	\begin{itemize}
		\item Sistema Operativo ArchCraft GNU Linux 64 bits Kernell
		\item NeoVim
		\item OpenJDK 64-Bit 20.0.1 
		\item Git 2.42.0
		\item Cuenta en GitHub con el correo institucional.
		\item Programación Orientada a Objetos.
		\item Creacion de programas con CLI	
            \item Bilioteca Graphics (origen propio)
	\end{itemize}
	\section{URL de Repositorio Github}
	\begin{itemize}
            \item URL del Repositorio GitHub para clonar o recuperar.
            \item \url{https://github.com/JhonatanDczel/fp2-23b.git}
            \item URL para el laboratorio \itemPracticeNumber{} en el Repositorio GitHub.
            \item \itemUrl
	\end{itemize}
