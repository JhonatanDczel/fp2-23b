\section{Ordenando un HashMap}
\begin{itemize}
  \item Antes de continuar, las especificaciones como, mostrar al soldado con mayor vida, mostrar el nivel global de vida y los datos por ejercito de todos los soldados ya han sido cubiertas desde muchos laboratorios anteriores, asi que no hay mayor cambio en su funcionamiento
  \item Asi que ahora nos centraremos en el ultimo cambio que se pide hacer en el laboratorio: el ranking de soldados por vida
  \item En anteriores laboratorios era demasiado simple implementar un metodo que una dos arrays de Soldados y luego aplicar un algoritmo de ordenamiento por vida
  \item Ahora las cosas se complican un poco mas, ya que los HashMap no tienen un orden especifico
  \item Para lograr trabajar con el HashMap, lo tendremos que convertir a un array, y para eso usamos el metodo:
\end{itemize}
\begin{lstlisting}
  public static Soldado[] toArray(HashMap<String, Soldado> armyH){
    Soldado[] army = new Soldado[armyH.size()];
    int i = 0;
    for(String name : armyH.keySet()){
      army[i] = armyH.get(name);
      i++;
    }
    return army;
  }
 
\end{lstlisting}
\begin{itemize}
  \item El codigo es simple, devuelve un array de Soldados, a partir de un HashMap
  \item Ahora con esto, podemos implementar otro metodo llamado ranking, que tomara dos arrays de Soldados como parametro, los unira, y los ordenara usando un algoritmo de ordenamiento:
\end{itemize}
\begin{lstlisting}
  public static HashMap<String, Soldado> ranking(HashMap<String, Soldado> army1, HashMap<String, Soldado> army2){
    Soldado[] a1 = toArray(army1);
    Soldado[] a2 = toArray(army2);
    Soldado[] total = new Soldado[a1.length + a2.length];
    int i = 0;

    for(Soldado s : a1){
      total[i] = s;
      i++;
    }
    for(Soldado s : a2){
      total[i] = s;
      i++;
    }
    bubbleSortLife(total);
    HashMap<String, Soldado> ranking = new HashMap<>();
    for(Soldado s : total){
      ranking.put(s.getName(), s);
    }
    return ranking;
  }
\end{lstlisting}
\begin{itemize}
  \item Como vemos, este metodo se apoya del metodo toArray que creamos anteriormente
  \item Con el metodo ranking ya creado, ahora podemos proceder a mostrar los datos por pantalla
\end{itemize}
