\section{Actividad principal}
A continuacion se cubriran los requerimientos principales de este laboratorio

\subsection{Iniciando 2 ejercitos}
    En este punto, se inicial dos ejercitos con nombres autogenerados, para eso, se modifica el codigo de inicializacion de ejercitos, para permitir ingresar un parametro extra (Un numero entero que indica el numero de ejercito)
\begin{lstlisting}[language=java, caption={Codigo fuente, Videojuego.java}]
public static void main(String[] args){
    Soldado[] army1 = initializeArmy(0, true); 
    Soldado[] army2 = initializeArmy(1, true); 
... 
  public static void displayArmy(Soldado[] army, String str){
    System.out.println("\n===== " + str + " =====");
    for(Soldado soldier : army){
      displaySoldier(soldier);
    }
  }
\end{lstlisting}
\begin{itemize}
  \item Como vemos, los ejercitos se generaran con un numero que sera lo que contendra su nombre
\end{itemize}
\subsection{Modificando los atributos en Soldado.java}

Se agrego un nuevo atributo en soldado, que contendra la coloracion del soldado, en este primer caso solo tendra dos opciones de color
\begin{itemize}
  \item Se agrego el atributo booleano negro
  \item Se agrego el metodo setter setNegro()
  \item Se agrego el metodo getter isNegro(), inicialmente se planteo llamarlo getNegro, pero al ser un valor booleano, y siguiendo las buenas practicas de programacion, se cambio a isNegro, para poder trabajar mas facilmente con su valor
\end{itemize}
\begin{lstlisting}[language=java, caption={Codigo fuente, Videojuego.java}]
public class Soldado{
  public String name;
  public int life;
  public int row;
  public int column;
  public boolean negro = false;

... 
... 
... 


  //SECCION DE SETERS

  public void setNegro(boolean n){
    this.negro = n;
  }

... 
... 
... 

  //SECCION DE GETERS

  public boolean isNegro(){
    return this.negro;
  }
\end{lstlisting}

\subsection{Implementacion del color del ejercito}
\begin{itemize}
  \item Una vez que ya tenemos una forma de distinguir un soldado, se pueden colorear de diferentes tonos
  \item Para eso se usa el metodo invertir(), incluido en graphics, que cambia o invierte el color de una ficha/pieza
  \item Al momento de inicializar un ejercito se tendra que especificar el color del mismo
\end{itemize}
\begin{lstlisting}[language=java, caption={Codigo fuente, Videojuego.java}]

  public static void makeGBoard(){
    for(int i = 0; i < 10; i++){
      Picture fila = null;
      for(int j = 0; j < 10; j++){
        Picture c = Picture.casilleroBlanco();
        if(board[i][j] != null){
          c = Picture.soldier().superponer(c);
          if(board[i][j].isNegro())
            c = Picture.soldier().invertir().superponer(c);
        }
\end{lstlisting}
\begin{itemize}
  \item En esa porcion de codigo, vemos el metodo que dibuja a los soldados por pantalla
  \item Se verifica si alguno tiene el metodo de isNegro como true, en caso de ser cierto, se invierte el color de un soldado y se agrega al tablero
\end{itemize}

\subsection{Muestra de datos por consola}
\begin{itemize}
  \item La parte final de la actividad es basicamente lo que hicimos por 3 laboratorios seguidos, mostrar los datos de vida, ordenar por algun criterio, generar un ranking de soldados y mostrar sus niveles de vida asi como el promedio y la vida mas grande
  \item Por lo tanto solo mostraremos el codigo sin explayarnos en su funcionamiento
\end{itemize}
\begin{lstlisting}[language=java, caption={Codigo fuente, Videojuego.java}]
  public static void main(String[] args){
    Soldado[] army1 = initializeArmy(0, true); 
    Soldado[] army2 = initializeArmy(1, true); 
    displayArmy(army1, "Ejercito 1");
    displayArmy(army2, "Ejercito 2");
    System.out.println("Soldado con maxima vida:");
    displaySoldier(maxLife);
    bubbleSortLife(army1);
    displayArmy(army1, "Ranking de soldados:");
    makeGBoard();
    displayBoard();
  }
\end{lstlisting}
\begin{itemize}
  \item El codigo pertenece al metodo principal
  \item Se inicializan dos ejercitos de soldados
  \item Se muestran por pantalla, gracias al metodo displayArmy, que mostrara sus niveles de vida, filas, columnas, nombres y puntos
  \item Se muestra el soldado de mayor vida, usando el metodo que hicimos dos laboratorios atras
  \item Se muestra el ranking de soldados, usando el metodo que une a los dos ejercitos, los ordena y muestra el resputado por pantalla
  \item Se utiliza el algoritmo de ordenamiento bubble sort para ordenar los ejercitos
  \item y finalmente se muestra el tablero gracias al metodo makeBoard y DisplayBoard que crea, y muestra el tablero respectivamente
\end{itemize}

