\section{Inicializando dos ejercitos}
\begin{itemize}
  \item Se necesitan crear dos ejercitos usando HashMap, para lo que crearemos un nuevo metodo que nos permita hacerlo
\end{itemize}
\begin{lstlisting} [language=java]
 public static HashMap<String, Soldado> initializeArmyHashMap(int n, boolean negro){

    int promLife = 0;
    Random rand = new Random();
    int randNum = rand.nextInt(10) + 1;
    HashMap<String, Soldado> army = new HashMap<>();

    for(int i = 0; i < randNum; i++){
      String nombre = "Soldado " + n + "x" + i;
      army.put(nombre,new Soldado(nombre));
      army.get(nombre).setNegro(negro);
      army.get(nombre).setLife(rand.nextInt(5) + 1);
      if(army.get(nombre).getLife() > maxLife.getLife())
        maxLife = army.get(nombre);
      promLife += army.get(nombre).getLife();
      genColumnRow(army.get(nombre));
    }
    promLife = promLife / army.size();
    promedio = (promLife + promedio) / 2;
    return army;

  }
\end{lstlisting}
\begin{itemize}
  \item El codigo es una adaptacion de la generacion normal de ejercitos en un array
  \item Con la diferencia de que los soldados creados estaran conforme en cantidad con el ultimo parametro que se le pase al metodo
  \item Los objetos Soldado se guardan como valores de las claves que son sus nombres
  \item Los otros dos parametros que recibe son n (numero identificador del ejercito) y negro (variable booleana que representa la coloracion de un ejercito)
  \item Para acceder a los Soldados y ponerlos se hacen uso de metodos de HashMap
    \item Se cumple con las especificaciones:
      \begin{itemize}
        \item Numero aleatorio entre 1 y 10
          \item Vida aleatoria entre 1 y 5
            \item Nombre autogenerado para cada uno  
              \item Que no hayan dos soldados en una misma casilla, esto se lograra con el siguiente metodo que situa a los soldados sobre el tablero:
      \end{itemize}
\end{itemize}
\begin{lstlisting} [language=java]
  public static void genColumnRow(Soldado s){
    Random rand = new Random();
    int column;
    int row;
    do {
      column = rand.nextInt(10);
      row = rand.nextInt(10);
    } while(!isEmpty(column, row));
    s.setColumn(column);
    s.setRow(row);
    board[row][column] = s;
  }
\end{lstlisting}
\begin{itemize}
  \item Este codigo consiste principalmente de un bucle do while 
  \item La condicion de parada es que se haya encontrado un sitio vacio para el lugar que se prueba en cada iteracion
  \item Se usa un metodo auxiliar que devuelve un valor de tipo booleano, es este:
\end{itemize}
\begin{lstlisting}
  public static boolean isEmpty(int column, int row){
    return board[row][column] == null;
  }
\end{lstlisting}
\begin{itemize}
  \item Este codigo verifica si el espacio en el que queremos insertar un objeto ya esta ocupado por otro
\end{itemize}
