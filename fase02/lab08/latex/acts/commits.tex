\section{Commits mas importantes}
\begin{itemize}
  \item A continuacion se muestran los commits mas importantes
\end{itemize}


    \begin{lstlisting}[language=bash, caption={commits mas importantes}]
    commit 53bc5c08d38fb9d931979bee8b1d013a615d7384
    Author: JhonatanDczel <jariasq@unsa.edu.pe>
    Date:   Wed Oct 18 11:54:46 2023 -0500

        Actividad 1: Crea un proyecto llamado Laboratorio7

        Se creo el directorio lab07 como subdirectorio de fase02

    commit 87e61fbcfd20fe0660ae218f3f324e23e5166329
    Author: JhonatanDczel <jariasq@unsa.edu.pe>
    Date:   Wed Oct 18 12:07:32 2023 -0500

        Actividad4: Regresa el modelo de ArrayList a array

        Siguiendo las indicaciones, se escogio el modelo de array bidimensional
        simple para el tablero, ya que el modo de trabajar con el es mas simple,
        la forma de acceder a elementos, modificarlos y crearlos es mucho mas
        rapido y eficiente que con un ArrayList.commit 94fbd5984ef78bba0f3279931da37cdb9
    ae225f7


    commit 94fbd5984ef78bba0f3279931da37cdb9ae225f7
    Author: JhonatanDczel <jariasq@unsa.edu.pe>
    Date:   Wed Oct 18 12:17:40 2023 -0500

        Actividad5: Inicia 2 ejercitos

        En este punto, se inicial dos ejercitos con nombres autogenerados, para
        eso, se modifica el codigo de inicializacion de jeercitos, para permitir
        ingresar un parametro extra (Un numero entero que indica el numero de
        ejercito)

    commit 5c63eb95e23f2878921d8c0e2ddae6b430c5df17
    Author: JhonatanDczel <jariasq@unsa.edu.pe>
    Date:   Wed Oct 18 12:20:37 2023 -0500

        Actividad5: Modifica los tributos en Soldado.java

        Se agrega un atributo llamado "negro" que es de valor booleano, es true
        en caso de querer que nuestro ejercito se muestre de color negro, y
        false, si queremos que nuestro ejercito se muestre en el color por
        defecto, blanco.

    commit da60e154a436ff0c1b4411386f323a14402edf6e
    Author: JhonatanDczel <jariasq@unsa.edu.pe>
    Date:   Wed Oct 18 12:35:23 2023 -0500

        Actividad 5: Implementa el cambio de color

        Se implemento el codigo en el metodo makeGBoard, para invertir el color
        en caso de que el atributo "isNegro" sea verdadero, para invertir el
        color se usa un metodo que esta en la biblioteca graphcis, que cambia la
        forma en que es representada la figura y luego cambia tambien el color
        en que es dibujada la figura


    commit da532908095e179fce9779e9963d13762d536b6a
    Author: JhonatanDczel <jariasq@unsa.edu.pe>
    Date:   Wed Oct 18 12:37:20 2023 -0500

        Actividad5: Prueba de coloreado

        Se hicieron 3 pruebas de coloreado de soldados, la primera con los
        soldados en blanco, la segunda alternando colores, y la tercera con los
        soldados de color

    \end{lstlisting}
