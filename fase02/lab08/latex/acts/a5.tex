\section{Metodo principal}
\begin{itemize}
  \item Ahora procedere a mostrar el metodo main que controla todas las acciones del programa
\end{itemize}
\begin{lstlisting}
  public static void main(String[] args){
    HashMap<String, Soldado> army1 = initializeArmyHashMap(0, false);
    HashMap<String, Soldado> army2 = initializeArmyHashMap(1, true);
    displayArmy(army1, "Ejercito 1");
    displayArmy(army2, "Ejercito 2");
    System.out.println("Soldado con maxima vida:");
    displaySoldier(maxLife);

    HashMap<String, Soldado> ranking = ranking(army1, army2);
    displayArmy(ranking, "Ranking de soldados:");
    makeGBoard();
    displayBoard();

  }
\end{lstlisting}
\begin{itemize}
  \item El metodo inicia generando dos HashMaps
  \item Luego muestra usando el metodo displayArmy que veremos a continuacion:
\end{itemize}
\begin{lstlisting}
  public static void displayArmy(HashMap<String, Soldado> army, String str){
    System.out.println("\n===== " + str + " =====");
    for(String soldado : army.keySet()){
      displaySoldier(army.get(soldado));
    }
  }

  public static void displaySoldier(Soldado s){
    System.out.println(" " + s.getName() + ":");
    System.out.println("  Nivel de vida: " + s.getLife());
    System.out.println("  Fila: " + (s.getRow() + 1));
    System.out.println("  Columna: " + (s.getColumn() + 1));
    System.out.print("\n");
  }
\end{lstlisting}
\begin{itemize}
  \item El metodo se apoya de otro, (displaySoldier) que muestra los datos de un soldado, y hace eso con todo el array de Soldados
  \item Continuando con el metodo principal, luego muestra al soldado con la mayor puntuacion de vida
  \item Luego se genera un nuevo HashMap "ranking" que tendra a los soldados ordenados
  \item Luego llama al metodo ranking e imprime su resultado
  \item Finalmente se crea el tablero grafico, y se muestra
  \item Tenemos las siguientes variables globales:
\end{itemize}
\begin{lstlisting}
  public static Soldado[][] board = new Soldado[10][10];
  public static Picture gBoard;
  public static Soldado maxLife = new Soldado("sold");
  public static int promedio = 0;
\end{lstlisting}

