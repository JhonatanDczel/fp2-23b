
\section{Ejercicio 01}

\subsection{Descripción General}
El código presentado para el primer ejercicio consta de dos clases en Java: \texttt{Punto} y \texttt{Circulo}. Estas clases se utilizan para representar puntos en un plano cartesiano y círculos respectivamente. La clase \texttt{Circulo} hereda de la clase \texttt{Punto}, lo que implica que un objeto de tipo \texttt{Circulo} hereda todas las propiedades y métodos de un objeto \texttt{Punto}.

\subsection{Clase \texttt{Punto}}
\subsubsection{Atributos}
\begin{itemize}
    \item \texttt{private double x}: Representa la coordenada x del punto.
    \item \texttt{private double y}: Representa la coordenada y del punto.
\end{itemize}

\subsubsection{Constructor}
\begin{itemize}
    \item \texttt{public Punto(double x, double y)}: Inicializa las coordenadas (x, y) del punto con los valores proporcionados.
\end{itemize}

\subsubsection{Métodos}
\begin{itemize}
    \item \texttt{public double distancia(Punto otroPunto)}: Calcula la distancia entre el punto actual y otro punto dado utilizando la fórmula de distancia euclidiana.
    \item \texttt{public double getX()}: Obtiene la coordenada x del punto.
    \item \texttt{public double getY()}: Obtiene la coordenada y del punto.
    \item \texttt{public void setY(double y)}: Establece la coordenada y del punto.
    \item \texttt{public void setX(double x)}: Establece la coordenada x del punto.
\end{itemize}

\subsection{Clase \texttt{Circulo}}
\subsubsection{Atributos}
\begin{itemize}
    \item \texttt{private double radio}: Representa el radio del círculo.
\end{itemize}

\subsubsection{Constructor}
\begin{itemize}
    \item \texttt{public Circulo(double x, double y, double radio)}: Inicializa las coordenadas (x, y) del centro del círculo y su radio. Utiliza la llamada al constructor de la clase base \texttt{Punto} mediante \texttt{super(x, y)}.
\end{itemize}

\subsubsection{Métodos}
\begin{itemize}
    \item \texttt{public double getRadio()}: Obtiene el radio del círculo.
    \item \texttt{public void setRadio(double radio)}: Establece el radio del círculo.
\end{itemize}

\subsection{Ejemplo de Uso}
\begin{lstlisting}[language=Java]
public class EjemploUso {
  public static void main(String[] args) {
    // Crear un punto en el plano cartesiano
    Punto puntoA = new Punto(1.0, 2.0);
    
    // Obtener las coordenadas del punto
    double coordenadaX = puntoA.getX();
    double coordenadaY = puntoA.getY();
    
    // Crear un círculo con centro en el puntoA y radio 3.0
    Circulo circuloA = new Circulo(coordenadaX, coordenadaY, 3.0);
    
    // Obtener el radio del círculo
    double radioCirculoA = circuloA.getRadio();
    
    // Establecer nuevas coordenadas al puntoA
    puntoA.setX(4.0);
    puntoA.setY(5.0);
    
    // Calcular la distancia entre el puntoA y el centro del círculoA
    double distancia = puntoA.distancia(new Punto(coordenadaX, coordenadaY));
  }
}
\end{lstlisting}

En este ejemplo, se muestran instancias de las clases \texttt{Punto} y \texttt{Circulo}, así como el acceso a sus métodos para obtener y establecer valores, y calcular la distancia entre el punto y el centro del círculo.
