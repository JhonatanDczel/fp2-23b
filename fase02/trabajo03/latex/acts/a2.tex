\section{Ejercicio 02}

\subsection{Descripción General}
Para el segundo ejercicio, se introduce la nueva clase: \texttt{Cilindro}. Además, se expande la clase \texttt{Circulo} que ahora hereda de la clase \texttt{Punto}. Estas clases están diseñadas para representar un cilindro tridimensional, puntos en un plano cartesiano, y círculos respectivamente.

\subsection{Clase Cilindro}
\textbf{Atributos:}
\begin{itemize}
    \item \texttt{protected double longitud}: Representa la longitud del cilindro.
\end{itemize}

\textbf{Constructor:}
\begin{itemize}
    \item \texttt{public Cilindro(double x, double y, double radio, double longitud)}: Inicializa las coordenadas \((x, y)\), el radio y la longitud del cilindro. Utiliza la llamada al constructor de la clase base \texttt{Circulo} mediante \texttt{super(x, y, radio)}.
\end{itemize}

\textbf{Métodos:}
\begin{itemize}
    \item \texttt{public double superficie()}: Calcula la superficie del cilindro.
    \item \texttt{public void setLongitud(double longitud)}: Establece la longitud del cilindro.
    \item \texttt{public double getLongitud()}: Obtiene la longitud del cilindro.
\end{itemize}

\subsection{Clase Punto (sin cambios)}
\textbf{Atributos:}
\begin{itemize}
    \item \texttt{protected double x}: Representa la coordenada x del punto.
    \item \texttt{protected double y}: Representa la coordenada y del punto.
\end{itemize}

\textbf{Constructor:}
\begin{itemize}
    \item \texttt{public Punto(double x, double y)}: Inicializa las coordenadas \((x, y)\) del punto con los valores proporcionados.
\end{itemize}

\textbf{Métodos:}
\begin{itemize}
    \item \texttt{public double distancia(Punto otroPunto)}: Calcula la distancia entre el punto actual y otro punto dado utilizando la fórmula de distancia euclidiana.
    \item \texttt{public double getX()}: Obtiene la coordenada x del punto.
    \item \texttt{public double getY()}: Obtiene la coordenada y del punto.
    \item \texttt{public void setY(double y)}: Establece la coordenada y del punto.
    \item \texttt{public void setX(double x)}: Establece la coordenada x del punto.
\end{itemize}

\subsection{Clase Circulo (sin cambios)}
\textbf{Atributos:}
\begin{itemize}
    \item \texttt{protected double radio}: Representa el radio del círculo.
\end{itemize}

\textbf{Constructor:}
\begin{itemize}
    \item \texttt{public Circulo(double x, double y, double radio)}: Inicializa las coordenadas \((x, y)\) y el radio del círculo. Utiliza la llamada al constructor de la clase base \texttt{Punto} mediante \texttt{super(x, y)}.
\end{itemize}

\textbf{Métodos:}
\begin{itemize}
    \item \texttt{public double getRadio()}: Obtiene el radio del círculo.
    \item \texttt{public void setRadio(double radio)}: Establece el radio del círculo.
\end{itemize}

\subsection{Ejemplo de Uso}
\begin{lstlisting}[language=Java]
public class EjemploUso {
  public static void main(String[] args) {
    // Crear un punto en el plano cartesiano
    Punto puntoA = new Punto(1.0, 2.0);
    
    // Obtener las coordenadas del punto
    double coordenadaX = puntoA.getX();
    double coordenadaY = puntoA.getY();
    
    // Crear un cilindro con centro en el puntoA, radio 3.0 y altura 10.0
    Cilindro cilindro = new Cilindro(coordenadaX, coordenadaY, 3.0, 10.0);
    
    // Obtener la superficie del cilindro
    double superficieCilindro = cilindro.superficie();
  }
}
\end{lstlisting}
