\section{Ejercicio 03}

\subsection{Descripción General}
En este ejercicio, se ha implementado un conjunto de interfaces y una clase para representar la funcionalidad de un hidroavión. Se han definido dos interfaces, \texttt{Barco} y \texttt{Avion}, cada una con un método que representa la acción específica de navegar y volar, respectivamente. Además, se ha creado una clase \texttt{Hidroavion} que implementa ambas interfaces, permitiendo que un objeto de esta clase pueda realizar tanto operaciones de barco como de avión.

\subsection{Código}
\textbf{Interfaces:}
\begin{lstlisting}[language=Java]
interface Barco {
  void navegar();
}

interface Avion {
  void volar();
}
\end{lstlisting}

\textbf{Clase Hidroavion:}
\begin{lstlisting}[language=Java]
class Hidroavion implements Barco, Avion {
  @Override
  public void navegar() {
    System.out.println("Hidroavion navegando en el agua");
  }

  @Override
  public void volar() {
    System.out.println("Hidroavion navegando en el aire");
  }
}
\end{lstlisting}

\textbf{Clase de Prueba (\texttt{Ejercicio3}):}
\begin{lstlisting}[language=Java]
public class Ejercicio3 {
  public static void main(String[] args) {
    // Crear una instancia de hidroavion
    Hidroavion hidroavion = new Hidroavion();

    // Llamar a los métodos de las interfaces
    hidroavion.navegar();
    hidroavion.volar();
  }
}
\end{lstlisting}

\subsection{Explicación del Código}
\textbf{Interfaces (\texttt{Barco} y \texttt{Avion}):}
\begin{itemize}
  \item Se han definido dos interfaces, una para la funcionalidad de barco y otra para la funcionalidad de avión. Cada interfaz contiene un método que representa la acción específica asociada con la respectiva interfaz.
\end{itemize}

\textbf{Clase \texttt{Hidroavion}:}
\begin{itemize}
  \item Esta clase implementa ambas interfaces \texttt{Barco} y \texttt{Avion}.
  \item Se han proporcionado las implementaciones concretas para los métodos \texttt{navegar} y \texttt{volar} en función del comportamiento esperado de un hidroavión.
\end{itemize}

\textbf{Clase de Prueba (\texttt{Ejercicio3}):}
\begin{itemize}
  \item En la clase de prueba, se crea una instancia de \texttt{Hidroavion}.
  \item Se invocan los métodos \texttt{navegar} y \texttt{volar} para demostrar la funcionalidad dual del hidroavión, capaz de navegar tanto en el agua como en el aire.
\end{itemize}

\subsection{Ejemplo de Ejecución}
Al ejecutar la clase \texttt{Ejercicio3}, se obtendrá la siguiente salida:
\begin{lstlisting}
Hidroavion navegando en el agua
Hidroavion navegando en el aire
\end{lstlisting}
Este resultado demuestra que el hidroavión es capaz de realizar ambas acciones, navegar en el agua y volar en el aire, gracias a la implementación de las interfaces \texttt{Barco} y \texttt{Avion}.
