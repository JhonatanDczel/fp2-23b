
	
	\vspace*{10pt}
	
	\begin{center}	
		\fontsize{17}{17} \textbf{ Informe de Laboratorio \itemPracticeNumber}
	\end{center}
	\centerline{\textbf{\Large Tema: \itemTheme}}
	\vspace*{0.5cm}	

	\begin{flushright}
		\begin{tabular}{|M{2.5cm}|N|}
			\hline 
			\rowcolor{tablebackground}
			\color{white} \textbf{Nota}  \\
			\hline 
			     \\[30pt]
			\hline 			
		\end{tabular}
	\end{flushright}	

	\begin{table}[H]
		\begin{tabular}{|x{4.9cm}|x{4.3cm}|x{5.1cm}|}
			\hline 
			\rowcolor{tablebackground}
			\color{white} \textbf{Estudiante} & \color{white}\textbf{Escuela}  & \color{white}\textbf{Asignatura}   \\
			\hline 
			{\itemStudent \par \itemEmail} & \itemSchool & {\itemCourse \par Semestre: \itemSemester \par Código: \itemCourseCode}     \\
			\hline 			
		\end{tabular}
	\end{table}		
	
	\begin{table}[H]
		\begin{tabular}{|x{4.7cm}|x{4.8cm}|x{4.8cm}|}
			\hline 
			\rowcolor{tablebackground}
			\color{white}\textbf{Laboratorio} & \color{white}\textbf{Tema}  & \color{white}\textbf{Duración}   \\
			\hline 
			\itemPracticeNumber & \itemTheme & 04 horas   \\
			\hline 
		\end{tabular}
	\end{table}
	
	\begin{table}[H]
		\begin{tabular}{|x{4.7cm}|x{4.8cm}|x{4.8cm}|}
			\hline 
			\rowcolor{tablebackground}
			\color{white}\textbf{Semestre académico} & \color{white}\textbf{Fecha de inicio}  & \color{white}\textbf{Fecha de entrega}   \\
			\hline 
			\itemAcademic & \itemInput &  \itemOutput  \\
			\hline 
		\end{tabular}
	\end{table}

	\section{Actividades}
	\begin{itemize}		
      \item Al ejecutar el videojuego, el programa deberá dar las opciones:
      \item 1. Juego rápido (tal cual como en el laboratorio 11)
Al acabar el juego mostrar las opciones de volver a jugar y de volver al
menú principal. También se deberá tener la posibilidad de cancelar el
juego actual en cualquier momento, permitiendo escoger entre empezar
un juego totalmente nuevo o salir al menú principal.
      \item 2. Juego personalizado: permite festionar ejércitos. Primero se generan los 2 ejércitos con sus respectivos soldados y se muestran sus datos. Luego se tendrá que escoger cuál de los 2 ejércitos se va a gestionar, después se mostrarán las siguientes opciones:
        \begin{itemize}
          \item Crear Soldado: permitirá crear un nuevo soldado personalizado
y añadir al final del ejército (recordar que límite es de 10
soldados por ejército)
          \item Eliminar Soldado (no debe permitir un ejército vacío)
          \item Clonar Soldado (crea una copia exacta del soldado) y se añade
al final del ejército (recordar que límite es de 10 soldados por
ejército)
          \item Modificar Soldado (con submenú para cambiar alguno de los
atributos nivelAtaque, nivelDefensa, vidaActual)
          \item Comparar Soldados (verifica si atributos: nombre, nivelAtaque,
nivelDefensa, vidaActual y vive son iguales)
          \item Intercambiar Soldados (intercambia 2 soldados en sus posiciones
en la estructura de datos del ejército)
          \item Ver soldado (Búsqueda por nombre)
          \item Ver ejército
          \item Sumar niveles (usando Method-Call Chaining), calcular las
sumatorias de nivelVida, nivelAtaque, nivelDefensa, velocidad de
todos los soldados de un ejército 
            \begin{itemize}
              \item Por ejemplo, si ejército tendría 3 soldados:
              \item s=s1.sumar(s2).sumar(s3);
              \item s es un objeto Soldado nuevo que contendría las
sumatorias de los 4 atributos indicados de los 3 soldados.
Ningún soldado cambia sus valores
            \end{itemize}
          \item Jugar (se empezará el juego con los cambios realizados) y con
las mismas opciones de la opción 1.
          \item Volver (muestra el menú principal)
Después de escoger alguna de las opciones 1) a 9) se podrá volver a
elegir uno de los ejércitos y se mostrarán las opciones 1) a 11)
        \end{itemize} 
        \item 3. Salir
	\end{itemize}
		
	\section{Equipos, materiales y temas utilizados}
	\begin{itemize}
		\item Sistema Operativo ArchCraft GNU Linux 64 bits Kernell
		\item NeoVim
		\item OpenJDK 64-Bit 20.0.1 
		\item Git 2.42.0
		\item Cuenta en GitHub con el correo institucional.
		\item Programación Orientada a Objetos.
		\item Creacion de programas con CLI	
            \item Bilioteca Graphics (origen propio)
	\end{itemize}
	\section{URL de Repositorio Github}
	\begin{itemize}
            \item URL del Repositorio GitHub para clonar o recuperar.
            \item \url{https://github.com/JhonatanDczel/fp2-23b.git}
            \item URL para el laboratorio \itemPracticeNumber{} en el Repositorio GitHub.
            \item \itemUrl
	\end{itemize}
