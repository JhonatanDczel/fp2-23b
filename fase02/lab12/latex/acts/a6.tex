\section{Métodos de gestión de Soldados}
Estas funciones proporcionan al usuario la capacidad de crear, eliminar, clonar y modificar soldados en su ejército, contribuyendo así a la personalización y estrategia dentro del juego. Cada función interactúa directamente con la matriz que representa el campo de juego y la lista de soldados del equipo correspondiente.

\subsection{Función \texttt{crearSoldado}}

\textbf{Propósito:} Permite al usuario crear un nuevo soldado para su ejército.

\textbf{Parámetros:}
\begin{itemize}
  \item \texttt{Soldado[][] t:} La matriz que representa el campo de juego.
  \item \texttt{ArrayList<Soldado> e:} Lista de soldados del equipo actual.
  \item \texttt{String team:} Símbolo del equipo actual.
\end{itemize}

\textbf{Descripción:}
\begin{lstlisting}[language=java]
public static void crearSoldado(Soldado[][] t, ArrayList<Soldado> e, String team) {
    Scanner sc = new Scanner(System.in);
    Soldado s = new Soldado(team);

    if(e.size() >= 10) {
      System.out.println("El ejercito tiene el maximo de soladados: ");
      return;
    }

    System.out.print("Ingresa el nivel del soldado: ");
    s.setNivelVida(sc.nextInt());
    System.out.print("Ingresa el nivel de ataque: ");
    s.setNivelAtaque(sc.nextInt());
    System.out.print("Ingresa el nivel de defensa: ");
    s.setNivelDefensa(sc.nextInt());
    System.out.print("Ingresa la columna: ");
    s.setColumna(sc.nextInt());
    System.out.print("Ingresa la fila: ");
    s.setFila(sc.nextInt());
    s.setNombre("Soldado " + (e.size() + 1));

    e.add(s);
    t[s.getColumna()][s.getFila()] = s;
 
}
\end{lstlisting}

\subsection{Función \texttt{eliminarSoldado}}

\textbf{Propósito:} Permite al usuario eliminar un soldado de su ejército.

\textbf{Parámetros:}
\begin{itemize}
  \item \texttt{Soldado[][] t:} La matriz que representa el campo de juego.
  \item \texttt{ArrayList<Soldado> e:} Lista de soldados del equipo actual.
  \item \texttt{String team:} Símbolo del equipo actual.
\end{itemize}

\textbf{Descripción:}
\begin{lstlisting}[language=java]
public static void eliminarSoldado(Soldado[][] t, ArrayList<Soldado> e, String team) {
    Scanner sc = new Scanner(System.in);

    if(e.size() <= 1) {
      System.out.println("El ejercito nesecita al menos 1 soldado");
      return;
    }

    System.out.print("Ingresa la columna: ");
    int col = sc.nextInt();
    System.out.print("Ingresa la fila: ");
    int fila = sc.nextInt();

    for(int i = 0; i < e.size(); i += 1) {
      if(e.get(i).getColumna() == col && e.get(i).getFila() == fila){
        e.remove(i);
        break;
      }
    }

    t[col][fila] = null;
 
}
\end{lstlisting}

\subsection{Función \texttt{clonarSoldado}}

\textbf{Propósito:} Permite al usuario clonar un soldado existente en una nueva posición.

\textbf{Parámetros:}
\begin{itemize}
  \item \texttt{Soldado[][] t:} La matriz que representa el campo de juego.
  \item \texttt{ArrayList<Soldado> e:} Lista de soldados del equipo actual.
  \item \texttt{String team:} Símbolo del equipo actual.
\end{itemize}

\textbf{Descripción:}
\begin{lstlisting}[language=java]
public static void clonarSoldado(Soldado[][] t, ArrayList<Soldado> e, String team) {
    Scanner sc = new Scanner(System.in);

    System.out.print("Ingresa la columna: ");
    int col = sc.nextInt();
    System.out.print("Ingresa la fila: ");
    int fila = sc.nextInt();

    Soldado soldadoOriginal = t[col][fila];

    Soldado soldadoNuevo = new Soldado(soldadoOriginal.getTeam());
    soldadoNuevo.setNivelAtaque(soldadoOriginal.getNivelAtaque());
    soldadoNuevo.setNivelDefensa(soldadoOriginal.getNivelDefensa());
    soldadoNuevo.setNivelVida(soldadoOriginal.getNivelVida());
    soldadoNuevo.setNombre("Soldado " + (e.size() + 1));

    System.out.print("Ingresa la columna nueva: ");
    int newCol = sc.nextInt();
    System.out.print("Ingresa la fila nueva: ");
    int newFila = sc.nextInt();

    e.add(soldadoNuevo);
    t[newCol][newFila] = soldadoNuevo;
}
\end{lstlisting}

\subsection{Función \texttt{modificarSoldado}}

\textbf{Propósito:} Permite al usuario modificar las características y posición de un soldado existente.

\textbf{Parámetros:}
\begin{itemize}
  \item \texttt{Soldado[][] t:} La matriz que representa el campo de juego.
  \item \texttt{ArrayList<Soldado> e:} Lista de soldados del equipo actual.
  \item \texttt{String team:} Símbolo del equipo actual.
\end{itemize}

\textbf{Descripción:}
\begin{lstlisting}[language=java]
public static void modificarSoldado(Soldado[][] t, ArrayList<Soldado> e, String team) {
    Scanner sc = new Scanner(System.in);

    System.out.print("Ingresa la columna: ");
    int col = sc.nextInt();
    System.out.print("Ingresa la fila: ");
    int fila = sc.nextInt();

    Soldado s = t[col][fila];
    System.out.println("Ingresa nivel de vida nuevo: ");
    s.setNivelVida(sc.nextInt());
    System.out.println("Ingresa nivel de ataque nuevo: ");
    s.setNivelAtaque(sc.nextInt());
    System.out.println("Ingresa nivel de defensa nuevo: ");
    s.setNivelDefensa(sc.nextInt());
    System.out.println("Ingresa la columna nueva: ");
    s.setColumna(sc.nextInt());
    System.out.println("Ingresa la fila nueva: ");
    s.setFila(sc.nextInt());
}
\end{lstlisting}

