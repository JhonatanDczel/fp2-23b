\section{Commits mas importantes}
\begin{itemize}
  \item A continuacion se muestran los commits mas importantes
  \item Los commits se hicieron siguiendo la convencion para commits de git, y siguiendo las recomendaciones practicas para hacer mensajes de commits
    \item Cada mensaje de commit esta estructurado por un titulo y una descripcion separados por una linea en blanco
\end{itemize}


\begin{lstlisting}[language=bash, caption={commits mas importantes}]

    commit c1082e4d2b725108cdbf1f4acb639a89da4174d8
    Author: JhonatanDczel <jariasq@unsa.edu.pe>
    Date:   Mon Oct 23 14:02:50 2023 -0500

        Creando el proyecto lab08

        Se copiaron los archivos soldado, videojuego y graphics del laboratorio
        anterior

    commit a362ecf01678492c2ba977fd8c3d4f866ccc0313
    Author: JhonatanDczel <jariasq@unsa.edu.pe>
    Date:   Mon Oct 23 14:12:17 2023 -0500

        Hace que el metodo initialice army devuelva hashmap

    commit 40035dcc9534d58e9b577105339203b1dd0aa839
    Author: JhonatanDczel <jariasq@unsa.edu.pe>
    Date:   Mon Oct 23 14:08:29 2023 -0500

        Creacion del metodo initialiceArmyHashMap

    commit c78508b8e703041b1d11d8410744bed4c9870f31
    Author: JhonatanDczel <jariasq@unsa.edu.pe>
    Date:   Mon Oct 23 14:25:05 2023 -0500

        Se cambio la forma de establecer el atributo negro

        Se cambio la forma en que se establece que un ejercito sea de color
        negro, antes del cambio, se usaba un condicional para evaluar el valor
        booleano negro, ahora, se ingresa directemente este valor como atributo
        del soldado

    commit 6e7b92190399f9550e32bb58243d0b07ba26d918
    Author: JhonatanDczel <jariasq@unsa.edu.pe>
    Date:   Mon Oct 23 15:06:45 2023 -0500

        Se crea metodo para ordenar el ejercito

        Como no se puede ordenar directamente un hashmap, ya que no mantiene un
        oprden especifico, se copia el contenido a un array antes de mandarlo al
        metodo de ordenamiento de soldados por vida

    commit edfb9c396ca95be5c3038fb18c4bb282894758da
    Author: JhonatanDczel <jariasq@unsa.edu.pe>
    Date:   Mon Oct 23 15:48:12 2023 -0500

        Adaptando el codigo para hacer el ordenamiento

        Se adapto el codigo para crear un array grande que una a los dos
        ejercitos, luego se creo el metodo ranking que agarra estos dos
        ejercitos, los junta y crea uno mas grande que ordena con el algoritmo
        bubble sort, finalmente el resultado se muestra en pantalla
\end{lstlisting}
