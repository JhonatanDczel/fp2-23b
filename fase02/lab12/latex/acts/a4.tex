\section{Juego Rapido}

El método \texttt{juegoRapido()} representa la implementación de un juego estratégico entre dos ejércitos de soldados dispuestos en una matriz bidimensional. A continuación, se proporciona una explicación detallada del código:

\subsection{Configuración Inicial}
La función comienza solicitando al usuario ingresar 'S' si desea salir de la partida.
Si se ingresa 'S', la función termina inmediatamente. En caso contrario, se procede a la configuración inicial del juego.
\begin{lstlisting}
Scanner sc = new Scanner(System.in);
System.out.println("Para salir de la partida ingresa 'S'");
String salir = sc.next();
if (salir.toLowerCase().charAt(0) == 's') return;
\end{lstlisting}

\subsection{Inicialización de Variables}
Se crea una matriz table de dimensiones 10x10 para representar el campo de batalla,
y dos listas (ejercito1 y ejercito2) que contendrán instancias de la clase Soldado.
La función fillTable() se llama para inicializar la matriz y las listas.
\begin{lstlisting}
Soldado[][] table = new Soldado[10][10];
ArrayList<Soldado> ejercito1 = new ArrayList<Soldado>();
ArrayList<Soldado> ejercito2 = new ArrayList<Soldado>();
fillTable(table, ejercito1, ejercito2);
\end{lstlisting}

\subsection{Impresión de Información Inicial}
Se imprime en consola el soldado con mayor nivel en cada ejército y,
posteriormente, se muestra el promedio de puntos de ambos ejércitos.
\begin{lstlisting}
printMayorNivel(table, ejercito1, ejercito2);
System.out.println("___________________________________");
System.out.println("IMPRIMIR PROMEDIO DE PUNTOS");
printPromedioPuntos(table, ejercito1, ejercito2);
System.out.println("___________________________________");
\end{lstlisting}

\subsection{Ordenamiento y Ranking}
Se imprime la lista de soldados ordenada por algún criterio no especificado
y se muestra el ranking de puntos de ambos ejércitos, utilizando algoritmos de ordenamiento Bubble Sort y Select Sort, respectivamente.
\begin{lstlisting}
System.out.println("___________________________________");
System.out.println("SOLDADOS ORDENADOS");
printSoladosOrdenados(table, ejercito1, ejercito2);
System.out.println("___________________________________");
System.out.println("RANKING DE PUNTOS EJERCITO1 POR BUBBLE SORT");
printRankingPointsBubble(table, ejercito1);
System.out.println("___________________________________");
System.out.println("RANKING DE PUNTOS EJERCITO2 POR SELECT SORT");
printRankingPointsSelect(table, ejercito2);
\end{lstlisting}

\subsection{Desarrollo del Juego}
Se inicia un bucle que representa los turnos del juego.
En cada turno, se muestra el estado actual del campo de batalla (printTable(table))
y se realiza un movimiento de los soldados. Se alterna entre los dos ejércitos en cada turno.
\begin{lstlisting}
int turno = 1;
while (true) {
  printTable(table);
  mover(table, ejercito1, ejercito2, turno);

  turno = turno == 1 ? 2 : 1;

  String winner = getWinner(ejercito1, ejercito2);

  if (winner != null) {
    System.out.println("###########################");
    System.out.println("GANA EL EQUIPO " + winner);
    System.out.println("###########################");
    break;
  }
}
\end{lstlisting}

\subsection{Conclusión del Juego}
Cuando se determina un ganador (mediante la función getWinner()),
se imprime el equipo ganador y se finaliza el juego.

Entonces, el método \texttt{juegoRapido()} encapsula la lógica de un juego estratégico entre dos ejércitos de soldados, proporcionando información inicial, estadísticas y una interfaz para que los usuarios participen en turnos alternativos hasta que se determine un ganador.
